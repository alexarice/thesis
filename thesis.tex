\documentclass[techreport]{cam-thesis}

\usepackage{packages}
\usepackage{macros}
\addbibresource{thesis.bib}

\title{A type-theoretic approach to semistrict higher categories}

%% The full name of the author (e.g.: James Smith):
\author{Alex Rice}

%% College affiliation:
\college{Darwin College}

%% College shield:
\collegeshield{CollegeShields/Darwin}

%% Submission date [optional]:
\submissiondate{TODO}

%% Declaration date:
\date{TODO}

%% PDF meta-info:
\subjectline{Computer Science}
\keywords{category theory, higher category theory, type theory}



%%%%%%%%%%%%%%%%%%%%%%%%%%%%%%%%%%%%%%%%%%%%%%%%%%%%%%%%%%%%%%%%%%%%%%%%%%%%%%%%
%% Abstract:
%%
\abstract{%
  Abstract to go here...
}


%%%%%%%%%%%%%%%%%%%%%%%%%%%%%%%%%%%%%%%%%%%%%%%%%%%%%%%%%%%%%%%%%%%%%%%%%%%%%%%%
%% Acknowledgements:
%%
\acknowledgements{%
  My acknowledgements ...
}


%%%%%%%%%%%%%%%%%%%%%%%%%%%%%%%%%%%%%%%%%%%%%%%%%%%%%%%%%%%%%%%%%%%%%%%%%%%%%%%%
%% Contents:
%%
\begin{document}

%%%%%%%%%%%%%%%%%%%%%%%%%%%%%%%%%%%%%%%%%%%%%%%%%%%%%%%%%%%%%%%%%%%%%%%%%%%%%%%%
%% Title page, abstract, declaration etc.:
%% -    the title page (is automatically omitted in the technical report mode).
\frontmatter{}



%%%%%%%%%%%%%%%%%%%%%%%%%%%%%%%%%%%%%%%%%%%%%%%%%%%%%%%%%%%%%%%%%%%%%%%%%%%%%%%%
%% Thesis body:
%%
\chapter{Introduction}
Things to go in introduction:
\begin{itemize}
\item Motivation for higher categories
\item Motivation for semistrictness
\item Introduce 3 classes of coherences
\item Ideas of semistrictness
\item Link to graphical ideas
\item Story behind Catt - HoTT - Brunerei
\item Link to Grothedieck higher cats
\end{itemize}

\chapter{Background}
\label{sec:background}

We begin with an overview of the important concepts required for the rest of the thesis. Throughout, we will assume knowledge of various basic concepts from computer science, as well as a basic knowledge of category theory (including functor categories, presheafs, and (co)limits). The following sections introduce a form of globular higher categories, present the necessary type theory prerequisites, and give a definition of the type theory \Catt, close to the original definition.

This section additionally serves as a place to introduce the various syntax and notation which will be used throughout the rest of the thesis.

\section{Higher categories}
\label{sec:higher-categories}

\todo[inline]{Add references throughout}
A higher category is a generalisation of the ordinary notion of a category to allow higher dimensional structure. This manifests in the form of allowing arrows or morphisms to have their source or target be another morphism instead of an object. More precisely, higher categories are equipped with the notion of an \(n\)-cell, where an \((n+1)\)-cell has source and target \(n\)-cells, and \(0\)-cells play the role of objects in an ordinary category.

The role of objects is played by \(0\)-cells, with \(1\)-cells as the morphisms between these objects. For \(0\)-cells \(x\) and \(y\), a \(1\)-cell \(f\) with source \(x\) and target \(y\) will be drawn as:
\[
  \begin{tikzcd}
    x & y
    \arrow["f", from=1-1, to=1-2]
  \end{tikzcd}
\]
or may be written as \(f: x \to y\).\todo{should arrowheads be smaller?} Two cells are \emph{parallel} if they have the same source and target. Between any two parallel \(n\)-cells \(f\) and \(g\), we have a set of \((n+1)\)-cells between them. A \(2\)-cell \(\alpha : f \to g\) may be drawn as:
\[
  \begin{tikzcd}
    x & y
    \arrow[""{name=0, anchor=center, inner sep=0}, "f", curve={height=-12pt}, from=1-1, to=1-2]
    \arrow[""{name=1, anchor=center, inner sep=0}, "g"', curve={height=12pt}, from=1-1, to=1-2]
    \arrow["\alpha", shorten <=3pt, shorten >=3pt, Rightarrow, from=1, to=0]
  \end{tikzcd}
\]
and a \(3\)-cell \(\gamma\) between parallel \(2\)-cells \(\alpha\) and \(\beta\) could be could be drawn as:
\[
  \begin{tikzcd}
    x && y
    \arrow[""{name=0, anchor=center, inner sep=0}, "f", curve={height=-15pt}, from=1-1, to=1-3]
    \arrow[""{name=1, anchor=center, inner sep=0}, "g"', curve={height=15pt}, from=1-1, to=1-3]
    \arrow[""{name=2, anchor=center, inner sep=0}, "\alpha", shift left=4, shorten <=3pt, shorten >=3pt, Rightarrow, from=1, to=0]
    \arrow[""{name=3, anchor=center, inner sep=0}, "\beta"', shift right=4, shorten <=3pt, shorten >=3pt, Rightarrow, from=1, to=0]
    \arrow["\gamma", shorten <=4pt, shorten >=4pt, Rightarrow, nfold=3, from=2, to=3]
  \end{tikzcd}
\]

Just as in ordinary \(1\)-category theory, we expect to be able to compose morphisms. For \(1\)-cells, nothing has changed, given \(1\)-cells \(f: x \to y\) and \(g : y \to z\) we form the composition \(f \star g\):
\[
  \begin{tikzcd}
    x & y & z
    \arrow[from=1-1, to=1-2, "f"]
    \arrow[from=1-2, to=1-3, "g"]
  \end{tikzcd}
\]
which has source \(x\) and target \(z\). We pause here to note that composition will be given in ``diagrammatic order'' throughout the whole thesis, which is the opposite of the order of function composition but the same as the order of the arrows if drawn head to tail. This is chosen as it will be common for us to draw higher dimensional arrows in a diagram, and rare for us to consider categories where the higher arrows are given by functions. In an attempt to avoid confusion, we use a star (\(\star\)) to represent composition of arrows or cells in a higher category, and will use a circle (\(\circ\)) only for function composition.

In two dimensions, there is no longer a singular composition operation. For \(2\)-cells \(\alpha : f \to g\) and \(\beta : g \to h\), the composite \(\alpha \star_1 \beta\) can be formed as before:
% https://q.uiver.app/#q=WzAsMixbMCwwLCJcXGJ1bGxldCJdLFsyLDAsIlxcYnVsbGV0Il0sWzAsMSwiZiIsMCx7ImN1cnZlIjotNH1dLFswLDEsImgiLDIseyJjdXJ2ZSI6NH1dLFswLDEsImciLDFdLFsyLDQsIlxcYWxwaGEiLDAseyJzaG9ydGVuIjp7InNvdXJjZSI6MjAsInRhcmdldCI6MjB9fV0sWzQsMywiXFxiZXRhIiwwLHsic2hvcnRlbiI6eyJzb3VyY2UiOjIwLCJ0YXJnZXQiOjIwfX1dXQ==
\[
  \begin{tikzcd}
    x && y
    \arrow[""{name=0, anchor=center, inner sep=0}, "f"', curve={height=24pt}, from=1-1, to=1-3]
    \arrow[""{name=1, anchor=center, inner sep=0}, "h", curve={height=-24pt}, from=1-1, to=1-3]
    \arrow[""{name=2, anchor=center, inner sep=0}, "g"{description}, from=1-1, to=1-3]
    \arrow["\alpha", shorten <=3pt, shorten >=3pt, Rightarrow, from=0, to=2]
    \arrow["\beta", shorten <=3pt, shorten >=3pt, Rightarrow, from=2, to=1]
  \end{tikzcd}
\]

We refer to this composition as \emph{vertical composition}. We can also compose cells \(\alpha\) and \(\beta\) in the following way:

% https://q.uiver.app/#q=WzAsMyxbMCwwLCJ4Il0sWzEsMCwieSJdLFsyLDAsInoiXSxbMCwxLCIiLDAseyJjdXJ2ZSI6LTN9XSxbMCwxLCIiLDIseyJjdXJ2ZSI6M31dLFsxLDIsIiIsMix7ImN1cnZlIjotM31dLFsxLDIsIiIsMix7ImN1cnZlIjozfV0sWzMsNCwiXFxhbHBoYSIsMCx7InNob3J0ZW4iOnsic291cmNlIjoyMCwidGFyZ2V0IjoyMH19XSxbNSw2LCJcXGJldGEiLDAseyJzaG9ydGVuIjp7InNvdXJjZSI6MjAsInRhcmdldCI6MjB9fV1d
\[
  \begin{tikzcd}
    x & y & z
    \arrow[""{name=0, anchor=center, inner sep=0}, "g", curve={height=-18pt}, from=1-1, to=1-2]
    \arrow[""{name=1, anchor=center, inner sep=0}, "f"', curve={height=18pt}, from=1-1, to=1-2]
    \arrow[""{name=2, anchor=center, inner sep=0}, "i", curve={height=-18pt}, from=1-2, to=1-3]
    \arrow[""{name=3, anchor=center, inner sep=0}, "h"', curve={height=18pt}, from=1-2, to=1-3]
    \arrow["\alpha", shorten <=5pt, shorten >=5pt, Rightarrow, from=1, to=0]
    \arrow["\beta", shorten <=5pt, shorten >=5pt, Rightarrow, from=3, to=2]
  \end{tikzcd}
\]

This composition is called the \emph{horizontal composition}, and is written \(\alpha \star_0 \beta\). The subscript refers to the dimension of the shared boundary in the composition, with the \(1\)-cell \(g\) being the shared boundary in the vertical composition example and the \(0\)-cell \(y\) being the shared boundary in the horizontal composition example. The dimension of this shared boundary is the \emph{codimension} of the composition.

This pattern continues with \(3\)-cells, which can be composed at codimension \(0\), \(1\), or \(2\), as below:

% https://q.uiver.app/#q=WzAsNyxbMiwwLCJcXGJ1bGxldCJdLFswLDAsIlxcYnVsbGV0Il0sWzMsMCwiXFxidWxsZXQiXSxbNSwwLCJcXGJ1bGxldCJdLFs2LDAsIlxcYnVsbGV0Il0sWzcsMCwiXFxidWxsZXQiXSxbOCwwLCJcXGJ1bGxldCJdLFsxLDAsIiIsMCx7ImN1cnZlIjotM31dLFsxLDAsIiIsMix7ImN1cnZlIjozfV0sWzIsMywiIiwwLHsiY3VydmUiOi00fV0sWzIsMywiIiwyLHsiY3VydmUiOjR9XSxbMiwzXSxbNCw1LCIiLDAseyJjdXJ2ZSI6LTN9XSxbNCw1LCIiLDIseyJjdXJ2ZSI6M31dLFs1LDYsIiIsMix7ImN1cnZlIjotM31dLFs1LDYsIiIsMix7ImN1cnZlIjozfV0sWzgsNywiIiwyLHsib2Zmc2V0IjotNSwic2hvcnRlbiI6eyJzb3VyY2UiOjIwLCJ0YXJnZXQiOjIwfX1dLFs4LDcsIiIsMCx7Im9mZnNldCI6NSwic2hvcnRlbiI6eyJzb3VyY2UiOjIwLCJ0YXJnZXQiOjIwfX1dLFs4LDcsIiIsMix7InNob3J0ZW4iOnsic291cmNlIjoyMCwidGFyZ2V0IjoyMH19XSxbMTAsMTEsIiIsMix7Im9mZnNldCI6LTQsInNob3J0ZW4iOnsic291cmNlIjoyMCwidGFyZ2V0IjoyMH19XSxbMTAsMTEsIiIsMCx7Im9mZnNldCI6NCwic2hvcnRlbiI6eyJzb3VyY2UiOjIwLCJ0YXJnZXQiOjIwfX1dLFsxMSw5LCIiLDEseyJvZmZzZXQiOi00LCJzaG9ydGVuIjp7InNvdXJjZSI6MjAsInRhcmdldCI6MjB9fV0sWzExLDksIiIsMSx7Im9mZnNldCI6NCwic2hvcnRlbiI6eyJzb3VyY2UiOjIwLCJ0YXJnZXQiOjIwfX1dLFsxMywxMiwiIiwyLHsib2Zmc2V0IjotMywic2hvcnRlbiI6eyJzb3VyY2UiOjIwLCJ0YXJnZXQiOjIwfX1dLFsxMywxMiwiIiwwLHsib2Zmc2V0IjozLCJzaG9ydGVuIjp7InNvdXJjZSI6MjAsInRhcmdldCI6MjB9fV0sWzE1LDE0LCIiLDIseyJvZmZzZXQiOi0zLCJzaG9ydGVuIjp7InNvdXJjZSI6MjAsInRhcmdldCI6MjB9fV0sWzE1LDE0LCIiLDAseyJvZmZzZXQiOjMsInNob3J0ZW4iOnsic291cmNlIjoyMCwidGFyZ2V0IjoyMH19XSxbMTYsMTgsIlxcZ2FtbWEiLDAseyJzaG9ydGVuIjp7InNvdXJjZSI6MjAsInRhcmdldCI6MjB9fV0sWzE4LDE3LCJcXGRlbHRhIiwwLHsic2hvcnRlbiI6eyJzb3VyY2UiOjIwLCJ0YXJnZXQiOjIwfX1dLFsyMSwyMiwiXFxnYW1tYSIsMCx7InNob3J0ZW4iOnsic291cmNlIjoyMCwidGFyZ2V0IjoyMH19XSxbMTksMjAsIlxcZGVsdGEiLDAseyJzaG9ydGVuIjp7InNvdXJjZSI6MjAsInRhcmdldCI6MjB9fV0sWzIzLDI0LCJcXGdhbW1hIiwwLHsic2hvcnRlbiI6eyJzb3VyY2UiOjIwLCJ0YXJnZXQiOjIwfX1dLFsyNSwyNiwiXFxkZWx0YSIsMCx7InNob3J0ZW4iOnsic291cmNlIjoyMCwidGFyZ2V0IjoyMH19XV0=
\[
  \begin{tikzcd}
    \bullet && \bullet & \bullet && \bullet & \bullet & \bullet & \bullet
    \arrow[""{name=0, anchor=center, inner sep=0}, curve={height=-18pt}, from=1-1, to=1-3]
    \arrow[""{name=1, anchor=center, inner sep=0}, curve={height=18pt}, from=1-1, to=1-3]
    \arrow[""{name=2, anchor=center, inner sep=0}, curve={height=-24pt}, from=1-4, to=1-6]
    \arrow[""{name=3, anchor=center, inner sep=0}, curve={height=24pt}, from=1-4, to=1-6]
    \arrow[""{name=4, anchor=center, inner sep=0}, from=1-4, to=1-6]
    \arrow[""{name=5, anchor=center, inner sep=0}, curve={height=-18pt}, from=1-7, to=1-8]
    \arrow[""{name=6, anchor=center, inner sep=0}, curve={height=18pt}, from=1-7, to=1-8]
    \arrow[""{name=7, anchor=center, inner sep=0}, curve={height=-18pt}, from=1-8, to=1-9]
    \arrow[""{name=8, anchor=center, inner sep=0}, curve={height=18pt}, from=1-8, to=1-9]
    \arrow[""{name=9, anchor=center, inner sep=0}, shift left=5, shorten <=5pt, shorten >=5pt, Rightarrow, from=1, to=0]
    \arrow[""{name=10, anchor=center, inner sep=0}, shift right=5, shorten <=5pt, shorten >=5pt, Rightarrow, from=1, to=0]
    \arrow[""{name=11, anchor=center, inner sep=0}, shorten <=5pt, shorten >=5pt, Rightarrow, from=1, to=0]
    \arrow[""{name=12, anchor=center, inner sep=0}, shift left=4, shorten <=3pt, shorten >=3pt, Rightarrow, from=3, to=4]
    \arrow[""{name=13, anchor=center, inner sep=0}, shift right=4, shorten <=3pt, shorten >=3pt, Rightarrow, from=3, to=4]
    \arrow[""{name=14, anchor=center, inner sep=0}, shift left=4, shorten <=3pt, shorten >=3pt, Rightarrow, from=4, to=2]
    \arrow[""{name=15, anchor=center, inner sep=0}, shift right=4, shorten <=3pt, shorten >=3pt, Rightarrow, from=4, to=2]
    \arrow[""{name=16, anchor=center, inner sep=0}, shift left=3, shorten <=5pt, shorten >=5pt, Rightarrow, from=6, to=5]
    \arrow[""{name=17, anchor=center, inner sep=0}, shift right=3, shorten <=5pt, shorten >=5pt, Rightarrow, from=6, to=5]
    \arrow[""{name=18, anchor=center, inner sep=0}, shift left=3, shorten <=5pt, shorten >=5pt, Rightarrow, from=8, to=7]
    \arrow[""{name=19, anchor=center, inner sep=0}, shift right=3, shorten <=5pt, shorten >=5pt, Rightarrow, from=8, to=7]
    \arrow["\gamma", shorten <=2pt, shorten >=2pt, Rightarrow, nfold=3, from=9, to=11]
    \arrow["\delta", shorten <=2pt, shorten >=2pt, Rightarrow, nfold=3, from=11, to=10]
    \arrow["\gamma", shorten <=3pt, shorten >=3pt, Rightarrow, nfold=3, from=14, to=15]
    \arrow["\delta", shorten <=3pt, shorten >=3pt, Rightarrow, nfold=3, from=12, to=13]
    \arrow["\gamma", shorten <=2pt, shorten >=2pt, Rightarrow, nfold=3, from=16, to=17]
    \arrow["\delta", shorten <=2pt, shorten >=2pt, Rightarrow, nfold=3, from=18, to=19]
  \end{tikzcd}
\]

For every \(n\)-cell \(x\), there is an \((n+1)\)-cell \(\id(x) : x \to x\), called the \emph{identity morphism}.

As with 1-categories, \(\infty\)-categories need to satisfy certain equalities, which fall into 3 groups: associativity, unitality, and interchange. The associativity laws are the same as for 1-categories, only now a law is needed for each composition (every dimension and codimension).

Unitality is again similar to the case for 1-categories, except we again need unitality laws for each composition. We note that for lower codimension compositions, an iterated identity is needed. For example given a \(2\)-cell \(\alpha : f \to g\), the appropriate equation for left unitality of horizontal composition is:
\[ \id(\id(x)) \star_0 \alpha = \alpha \]
In general for a unit to be cancelled, it must be iterated a number of times equal to the difference between the dimension and codimension of the composition.

Interchange laws do not appear in 1-categories, and specify how compositions of different dimensions interact. The first interchange law states that for suitable \(2\)-cells \(\alpha\), \(\beta\), \(\gamma\), and \(\delta\), that:
\[ (\alpha \star_0 \gamma) \star_1 (\beta \star_0 \delta) = (\alpha \star_1 \beta) \star_0 ()\]
This can be diagrammatically depicted as:
\newsavebox{\innertop}
\savebox{\innertop}{
  \adjustbox{scale=0.8}{\begin{tikzcd}[ampersand replacement=\&,column sep=small]
    \bullet \& \bullet \& \bullet
    \arrow[""{name=0, anchor=center, inner sep=0}, curve={height=-12pt}, from=1-1, to=1-2]
    \arrow[""{name=1, anchor=center, inner sep=0}, curve={height=12pt}, from=1-1, to=1-2]
    \arrow[""{name=2, anchor=center, inner sep=0}, curve={height=-12pt}, from=1-2, to=1-3]
    \arrow[""{name=3, anchor=center, inner sep=0}, curve={height=12pt}, from=1-2, to=1-3]
    \arrow["\alpha", shorten <=3pt, shorten >=3pt, Rightarrow, from=1, to=0]
    \arrow["\gamma", shorten <=3pt, shorten >=3pt, Rightarrow, from=3, to=2]
  \end{tikzcd}}}
\newsavebox{\innerbot}
\savebox{\innerbot}{
  \adjustbox{scale=0.8}{\begin{tikzcd}[ampersand replacement=\&,column sep=small]
    \bullet \& \bullet \& \bullet
    \arrow[""{name=0, anchor=center, inner sep=0}, curve={height=-12pt}, from=1-1, to=1-2]
    \arrow[""{name=1, anchor=center, inner sep=0}, curve={height=12pt}, from=1-1, to=1-2]
    \arrow[""{name=2, anchor=center, inner sep=0}, curve={height=-12pt}, from=1-2, to=1-3]
    \arrow[""{name=3, anchor=center, inner sep=0}, curve={height=12pt}, from=1-2, to=1-3]
    \arrow["\beta", shorten <=3pt, shorten >=3pt, Rightarrow, from=1, to=0]
    \arrow["\delta", shorten <=3pt, shorten >=3pt, Rightarrow, from=3, to=2]
  \end{tikzcd}}}
\newsavebox{\innerleft}
\savebox{\innerleft}{
  \adjustbox{scale=1}{\begin{tikzcd}[ampersand replacement=\&,column sep=small,cramped]
    \bullet \& \bullet
    \arrow[""{name=0, anchor=center, inner sep=0}, controls=+(80:0.7) and +(100:0.7),, from=1-1, to=1-2]
    \arrow[""{name=1, anchor=center, inner sep=0}, curve={height=0}, from=1-1, to=1-2]
    \arrow[""{name=2, anchor=center, inner sep=0}, controls=+(100:-0.7) and +(80:-0.7),, from=1-1, to=1-2]
    \arrow["\alpha", shorten <=3pt, shorten >=3pt, Rightarrow, from=2, to=1]
    \arrow["\beta", shorten <=3pt, shorten >=3pt, Rightarrow, from=1, to=0]
  \end{tikzcd}}}
\newsavebox{\innerright}
\savebox{\innerright}{
  \adjustbox{scale=1}{\begin{tikzcd}[ampersand replacement=\&,column sep=small,cramped]
    \bullet \& \bullet
    \arrow[""{name=0, anchor=center, inner sep=0}, controls=+(80:0.7) and +(100:0.7), from=1-1, to=1-2]
    \arrow[""{name=1, anchor=center, inner sep=0}, curve={height=0}, from=1-1, to=1-2]
    \arrow[""{name=2, anchor=center, inner sep=0}, controls=+(100:-0.7) and +(80:-0.7),, from=1-1, to=1-2]
    \arrow["\gamma", shorten <=3pt, shorten >=3pt, Rightarrow, from=2, to=1]
    \arrow["\delta", shorten <=3pt, shorten >=3pt, Rightarrow, from=1, to=0]
  \end{tikzcd}}}
\[
  \begin{tikzcd}[column sep=small]
    \bullet &&&&& \bullet & {=} & \bullet &&& \bullet &&& \bullet
    \arrow[""{name=0, anchor=center, inner sep=0}, from=1-1, to=1-6]
    \arrow[""{name=1, anchor=center, inner sep=0}, draw=none, controls=+(90:2) and +(90:2), from=1-1, to=1-6]
    \arrow[""{name=2, anchor=center, inner sep=0}, draw=none, controls=+(90:-2) and +(90:-2), from=1-1, to=1-6]
    \arrow[""{name=4, anchor=center, inner sep=0}, draw=none, controls=+(80:1.5) and +(100:1.5), from=1-8, to=1-11]
    \arrow[""{name=5, anchor=center, inner sep=0}, draw=none, controls=+(100:-1.5) and +(80:-1.5), from=1-8, to=1-11]
    \arrow[""{name=6, anchor=center, inner sep=0}, draw=none, controls=+(80:1.5) and +(100:1.5), from=1-11, to=1-14]
    \arrow[""{name=8, anchor=center, inner sep=0}, draw=none, controls=+(100:-1.5) and +(80:-1.5), from=1-11, to=1-14]
    \arrow["\usebox{\innertop}"{description, inner sep = 0,xshift = -1.2pt}, shorten <=4pt, shorten >=4pt, Rightarrow, from=2, to=0]
    \arrow["\usebox{\innerbot}"{description, inner sep = 0,xshift = -1.2pt}, shorten <=4pt, shorten >=4pt, Rightarrow, from=0, to=1]
    \arrow[""{name=1, anchor=center, inner sep=0}, controls=+(90:2) and +(90:2), from=1-1, to=1-6]
    \arrow[""{name=2, anchor=center, inner sep=0}, controls=+(90:-2) and +(90:-2), from=1-1, to=1-6]
    \arrow["\usebox{\innerleft}"{description, inner sep = 0,xshift = -1.3pt}, shorten <=2pt, shorten >=2pt, Rightarrow, from=5, to=4]
    \arrow["\usebox{\innerright}"{description, inner sep = 0,xshift = -1.3pt}, shorten <=2pt, shorten >=2pt, Rightarrow, from=8, to=6]
    \arrow[controls=+(80:1.5) and +(100:1.5), from=1-8, to=1-11]
    \arrow[controls=+(100:-1.5) and +(80:-1.5), from=1-8, to=1-11]
    \arrow[controls=+(80:1.5) and +(100:1.5), from=1-11, to=1-14]
    \arrow[controls=+(100:-1.5) and +(80:-1.5), from=1-11, to=1-14]
  \end{tikzcd}
\]
% \[
%   \begin{tikzcd}[row sep = small]
%     \bullet & \bullet & \bullet \\
%     & {\star_1} \\
%     \bullet & \bullet & \bullet
%     \arrow[""{name=0, anchor=center, inner sep=0}, curve={height=-18pt}, from=1-1, to=1-2]
%     \arrow[""{name=1, anchor=center, inner sep=0}, curve={height=18pt}, from=1-1, to=1-2]
%     \arrow[""{name=2, anchor=center, inner sep=0}, curve={height=-18pt}, from=1-2, to=1-3]
%     \arrow[""{name=3, anchor=center, inner sep=0}, curve={height=18pt}, from=1-2, to=1-3]
%     \arrow[""{name=4, anchor=center, inner sep=0}, curve={height=-18pt}, from=3-1, to=3-2]
%     \arrow[""{name=5, anchor=center, inner sep=0}, curve={height=18pt}, from=3-1, to=3-2]
%     \arrow[""{name=6, anchor=center, inner sep=0}, curve={height=-18pt}, from=3-2, to=3-3]
%     \arrow[""{name=7, anchor=center, inner sep=0}, curve={height=18pt}, from=3-2, to=3-3]
%     \arrow["\alpha", shorten <=5pt, shorten >=5pt, Rightarrow, from=5, to=4]
%     \arrow["\gamma", shorten <=5pt, shorten >=5pt, Rightarrow, from=7, to=6]
%     \arrow["\beta", shorten <=5pt, shorten >=5pt, Rightarrow, from=1, to=0]
%     \arrow["\delta", shorten <=5pt, shorten >=5pt, Rightarrow, from=3, to=2]
%   \end{tikzcd}
%   \quad=\quad
%   \begin{tikzcd}
%     \bullet & \bullet
%     \arrow[""{name=0, anchor=center, inner sep=0}, curve={height=-30pt}, from=1-1, to=1-2]
%     \arrow[""{name=1, anchor=center, inner sep=0}, curve={height=30pt}, from=1-1, to=1-2]
%     \arrow[""{name=2, anchor=center, inner sep=0}, from=1-1, to=1-2]
%     \arrow["\alpha"', shorten <=4pt, shorten >=4pt, Rightarrow, from=1, to=2]
%     \arrow["\beta"', shorten <=4pt, shorten >=4pt, Rightarrow, from=2, to=0]
%   \end{tikzcd}
%   \star_0
%   \begin{tikzcd}
%     \bullet & \bullet
%     \arrow[""{name=0, anchor=center, inner sep=0}, curve={height=-30pt}, from=1-1, to=1-2]
%     \arrow[""{name=1, anchor=center, inner sep=0}, curve={height=30pt}, from=1-1, to=1-2]
%     \arrow[""{name=2, anchor=center, inner sep=0}, from=1-1, to=1-2]
%     \arrow["\gamma"', shorten <=4pt, shorten >=4pt, Rightarrow, from=1, to=2]
%     \arrow["\delta"', shorten <=4pt, shorten >=4pt, Rightarrow, from=2, to=0]
%   \end{tikzcd}
% \]
There are also interchange laws for the interaction of composition and identities; A composition of two identities is the same as an identity on the composition of the underlying cells.


The \(\infty\)-categories that we study in this thesis will be globular, meaning that their cells form a globular set. A globular set can be seen as natural extension of the data of a category, whose data can be arranged into the following diagram:
% https://q.uiver.app/#q=WzAsMixbMCwwLCJZIl0sWzEsMCwiWCJdLFswLDEsInMiLDAseyJvZmZzZXQiOi0xfV0sWzAsMSwidCIsMix7Im9mZnNldCI6MX1dXQ==
\[
  \begin{tikzcd}
    M & O
    \arrow["s", shift left, from=1-1, to=1-2]
    \arrow["t"', shift right, from=1-1, to=1-2]
  \end{tikzcd}
\]
where \(O\) is a set of objects, \(M\) is a set of all morphisms, and \(s\) and \(t\) are functions assigning each morphism to its source and target object respectively. \(2\)-cells can be added to this diagram in a natural way:

% https://q.uiver.app/#q=WzAsMyxbMSwwLCJDXzEiXSxbMiwwLCJDXzAiXSxbMCwwLCJDXzIiXSxbMCwxLCJzXzAiLDAseyJvZmZzZXQiOi0xfV0sWzAsMSwidF8wIiwyLHsib2Zmc2V0IjoxfV0sWzIsMCwic18xIiwwLHsib2Zmc2V0IjotMX1dLFsyLDAsInRfMSIsMix7Im9mZnNldCI6MX1dXQ==
\[
  \begin{tikzcd}
    {C_2} & {C_1} & {C_0}
    \arrow["{s_0}", shift left, from=1-2, to=1-3]
    \arrow["{t_0}"', shift right, from=1-2, to=1-3]
    \arrow["{s_1}", shift left, from=1-1, to=1-2]
    \arrow["{t_1}"', shift right, from=1-1, to=1-2]
  \end{tikzcd}
\]
In a globular set, the source and target of any cell must be parallel, meaning they share the same source and target. This condition is imposed by \emph{globularity conditions}. Adding these and iterating the process leads to the following definition.

\begin{definition}
  The category of globes \(\mathbf{G}\) has objects given by the natural numbers and morphisms generated from \(\mathbf{s}_n, \mathbf{t}_n : n \to n + 1\) quotiented by the \emph{globularity conditions}:
  \begin{align*}
    \mathbf{s}_{n+1} \circ \mathbf{s}_n &= \mathbf{t}_{n+1} \circ \mathbf{s}_n\\
    \mathbf{s}_{n+1} \circ \mathbf{t}_n &= \mathbf{t}_{n+1} \circ \mathbf{t}_n\\
  \end{align*}

   The category of globular sets \(\mathbf{Glob}\), is the presheaf category \([\mathbf{G}, \mathbf{Set}]\).
 \end{definition}

Unwrapping this definition, a globular set \(G\) consists of sets \(G(n)\) for each \(n \in \mathbb{N}\), with source and target maps \(s_n, t_n : G(n+1) \to G(n)\), forming the following diagram:
\[
  \begin{tikzcd}
    \cdots & {G(3)} & {G(2)} & {G(1)} & {G(0)}
    \arrow["{s_0}", shift left, from=1-4, to=1-5]
    \arrow["{t_0}"', shift right, from=1-4, to=1-5]
    \arrow["{s_1}", shift left, from=1-3, to=1-4]
    \arrow["{t_1}"', shift right, from=1-3, to=1-4]
    \arrow["{t_2}"', shift right, from=1-2, to=1-3]
    \arrow["{s_2}", shift left, from=1-2, to=1-3]
    \arrow[shift right, from=1-1, to=1-2]
    \arrow[shift left, from=1-1, to=1-2]
  \end{tikzcd}
\]
and satisfying the globularity conditions. A morphism of globular sets \(F : G \to H\) is a collection of functions \(G(n) \to H(n)\) which commute with source and target maps.

Given a globular set \(G\), we will call the elements of \(G(n)\) the \(n\)-cells and write \(f : x \to y\) for an \((n+1)\)-cell \(f\) where \(s_n(f) = x\) and \(t_n(f) = y\). We further define the \(n\)-boundary operators \(\delta_n^-\) and \(\delta_n^+\) which take the source or target respectively of a \((n+k)\)-cell \(k\) times, returning an \(n\)-cell.

\begin{example}
  \label{ex:disc}
  The \(n\)-disc \(D^n\) is a finite globular set given by \(Y(n)\), where \(Y\) is the Yoneda functor \(\mathbf{G} \to \mathbf{Glob}\). \(D^n\), has no \(k\)-cells for \(k > n\), a single \(n\)-cell \(d_n\), and two \(m\)-cells \(d_m^-\) and \(d_m^+\) for \(m < n\). Every \((m+1)\)-cell of \(D^n\) has source \(d_m^-\) and target \(d_m^+\). The first few discs are depicted in \cref{fig:discs}. The Yoneda lemma tells us that a map of globular sets \(D^n \to G\) is the same as an \(n\)-cell of \(G\). For an \(n\)-cell \(x\) of \(G\), we let \(\{x\}\) be the unique map \(D^n \to G\) which sends \(d_n\) to \(x\).
\end{example}

\begin{figure}[h]
  \centering
  \begin{tabular}{P{3cm} P{3cm} P{3cm} P{3cm}}
    \(D^0\)&\(D^1\)&\(D^2\)&\(D^3\)\\
    {\begin{tikzcd}
        d_0
      \end{tikzcd}
    }&{\begin{tikzcd}[ampersand replacement=\&]
        d_0^- \& d_0^+
        \arrow[from=1-1, to=1-2, "d_1"]
      \end{tikzcd}
       }&{\begin{tikzcd}[ampersand replacement=\&]
           d_0^- \& d_0^+
           \arrow[""{name=0, anchor=center, inner sep=0}, "d_1^+", curve={height=-18pt}, from=1-1, to=1-2]
           \arrow[""{name=1, anchor=center, inner sep=0}, "d_1^-"', curve={height=18pt}, from=1-1, to=1-2]
           \arrow["d_2", shorten <=3pt, shorten >=3pt, Rightarrow, from=1, to=0]
         \end{tikzcd}
          }&{\begin{tikzcd}[ampersand replacement=\&]
              d_0^- \&\& d_0^+
              \arrow[""{name=0, anchor=center, inner sep=0}, "d_1^+", curve={height=-25pt}, from=1-1, to=1-3]
              \arrow[""{name=1, anchor=center, inner sep=0}, "d_1^-"', curve={height=25pt}, from=1-1, to=1-3]
              \arrow[""{name=2, anchor=center, inner sep=0}, "d_2^-", shift left=12pt,Rightarrow, shorten <=5pt, shorten >=5pt, from=1,to=0]
              \arrow[""{name=3, anchor=center, inner sep=0}, "d_2^+"', shift right=12pt,Rightarrow, shorten <=5pt, shorten >=5pt, from=1,to=0]
              \arrow["d_3", Rightarrow, nfold = 3, shorten <=3pt, shorten >=3pt,from=2,to=3]
            \end{tikzcd}}
  \end{tabular}
  \caption{The first disc globular sets}
  \label{fig:discs}
\end{figure}

We can now give the definition of a (strict) \(\infty\)-category.

\begin{definition}
  A (strict) \(\infty\)-category is a globular set \(G\) with operations:
  \begin{itemize}
  \item For \(m < n\), a composition \(\star_m\) taking \(n\)-cells \(f\) and \(g\) with \(\delta_m^+(f) = \delta_m^-(g)\) and giving an \(n\)-cell \(f \star_m g\) with:
    \begin{align*}
      s(f \star_m g) &= \begin{cases*}
        s(f)&\text{if \(m = n - 1\)}\\
        s(f) \star_m s(g)&\text{otherwise}
      \end{cases*}\\
      t(f \star_m g) &= \begin{cases*}
        t(g)&\text{if \(m = n - 1\)}\\
        t(f) \star_m t(g)&\text{otherwise}
      \end{cases*}
    \end{align*}
  \item For \(n\)-cell \(x\), an identity \((n+1)\)-cell \(\id(x) : x \to x\).
  \end{itemize}
  and satisfying equalities:
  \begin{itemize}
  \item Associativity: Given \(m < n\) and \(n\)-cells \(f\), \(g\), and \(h\) with \(\delta_m^+(f) = \delta_m^-(g)\) and \(\delta_m^+(g) = \delta_m^-(h)\):
    \[ (f \star_m g) \star_m h = f \star_m (g \star_m h) \]
  \item Unitality: Given \(m < n\) and \(n\)-cell \(f\):
    \begin{align*}
      \id^{n-m}(\delta_m^-(f)) \star_m f &= f\\
      f \star_m \id^{n-m}(\delta_m^+(f)) &= f
    \end{align*}
  \item Composition interchange: If \(o < m < n\) and \(\alpha\), \(\beta\), \(\gamma\), and \(\delta\) be \(n\)-cells with
    \[\delta_m^+(\alpha) = \delta_m^-(\beta)\qquad
      \delta_m^+(\gamma) = \delta_m^-(\delta)\qquad
      \delta_o^+(\alpha) = \delta_o^-(\gamma)\]
    then:
    \[(\alpha \star_o \gamma) \star_m (\beta \star_o \delta) = (\alpha \star_m \beta) \star_o (\gamma \star_m \delta)\]
  \item Identity interchange: Let \(m < n\) and \(f\) and \(g\) be \(n\)-cells with \(\delta_m^+(f) = \delta_m^-(g)\). Then:
    \[\id(f) \star_m \id(g) = \id(f \star_m g)\]
  \end{itemize}
  A morphism of \(\infty\) categories is a morphism of the underlying globular sets which preserves composition and identities.
\end{definition}

There is a clear forgetful functor from the category of strict infinity categories to the category of globular sets, which has a left adjoint given by taking the free strict infinity category over a globular set. \todo{Do i need monad?}

We end this section with an example of a non-trivial application of the axioms of an infinity category, known as the Eckmann-Hilton argument. The argument show's that any two scalars (morphisms from the identity to the identity) commute.

\begin{proposition}[Eckmann-Hilton]
  \label{prop:eh}
  Let \(x\) be an \(n\)-cell in an \(\infty\)-category and let \(\alpha\) and \(\beta\) be \((n+2)\)-cells with source and target \(\id(x)\). Then \(\alpha \star_{n+1} \beta = \beta \star_{n+1} \alpha\).
\end{proposition}
\begin{proof}
  The cells \(\alpha\) and \(\beta\) can be manoeuvred around each other as follows:
  \begin{align*}
    &\phantom{{}={}} \alpha \star_{n+1} \beta \\
    &= (\alpha \star_n i) \star_{n+1} (i \star_n \beta)&\text{Unitality}\\
    &= (\alpha \star_{n+1} i) \star_n (i \star_{n+1} \beta)&\text{Interchange}\\
    &= \alpha \star_n \beta &\text{Unitality}\\
    &= (i \star_{n+1} \alpha) \star_n (\beta \star_{n+1} i)&\text{Unitality}\\
    &= (i \star_n \beta) \star_{n+1} (\alpha \star_n i)&\text{Interchange}\\
    &= \beta \star_{n+1} \alpha&\text{Unitality}
  \end{align*}
  Where \(i = \id(\id(x))\).
\end{proof}

We give a more graphical representation of the proof in \cref{fig:eh}. In this proof the \(\alpha\) is moved to the left of \(\beta\), though we equally could have moved it round the right, and the choice made was arbitrary.

\newsavebox{\ehalpha}
\savebox{\ehalpha}{\adjustbox{scale=0.8}{
  \begin{tikzcd}[ampersand replacement=\&,column sep=small,cramped]
    \bullet \& \bullet \& \bullet
    \arrow[""{name=0, anchor=center, inner sep=0}, curve={height=-10pt}, from=1-1, to=1-2]
    \arrow[""{name=1, anchor=center, inner sep=0}, curve={height=10pt}, from=1-1, to=1-2]
    \arrow[""{name=2, anchor=center, inner sep=0}, curve={height=-10pt}, from=1-2, to=1-3]
    \arrow[""{name=3, anchor=center, inner sep=0}, curve={height=10pt}, from=1-2, to=1-3]
    \arrow["\alpha"', color={rgb,255:red,0;green,24;blue,204}, shorten <=3pt, shorten >=3pt, Rightarrow, from=1, to=0]
    \arrow["\id"', shorten <=3pt, shorten >=3pt, Rightarrow, from=3, to=2]
  \end{tikzcd}}}
\newsavebox{\ehbeta}
\savebox{\ehbeta}{\adjustbox{scale=0.8}{
  \begin{tikzcd}[ampersand replacement=\&,column sep=small,cramped]
    \bullet \& \bullet \& \bullet
    \arrow[""{name=0, anchor=center, inner sep=0}, curve={height=-10pt}, from=1-1, to=1-2]
    \arrow[""{name=1, anchor=center, inner sep=0}, curve={height=10pt}, from=1-1, to=1-2]
    \arrow[""{name=2, anchor=center, inner sep=0}, curve={height=-10pt}, from=1-2, to=1-3]
    \arrow[""{name=3, anchor=center, inner sep=0}, curve={height=10pt}, from=1-2, to=1-3]
    \arrow["\id"',  shorten <=3pt, shorten >=3pt, Rightarrow, from=1, to=0]
    \arrow["\beta"', color={rgb,255:red,204;green,0;blue,14}, shorten <=3pt, shorten >=3pt, Rightarrow, from=3, to=2]
  \end{tikzcd}}}
\newsavebox{\ehlefttop}
\savebox{\ehlefttop}{
  \adjustbox{scale=1}{\begin{tikzcd}[ampersand replacement=\&,column sep=small,cramped]
     \bullet \& \bullet
     \arrow[""{name=0, anchor=center, inner sep=0}, controls=+(80:0.7) and +(100:0.7),, from=1-1, to=1-2]
     \arrow[""{name=1, anchor=center, inner sep=0}, curve={height=0}, from=1-1, to=1-2]
     \arrow[""{name=2, anchor=center, inner sep=0}, controls=+(100:-0.7) and +(80:-0.7),, from=1-1, to=1-2]
     \arrow["\alpha", color= blue, shorten <=3pt, shorten >=3pt, Rightarrow, from=2, to=1]
     \arrow["\id", shorten <=3pt, shorten >=3pt, Rightarrow, from=1, to=0]
   \end{tikzcd}}}
\newsavebox{\ehrighttop}
\savebox{\ehrighttop}{
  \adjustbox{scale=1}{\begin{tikzcd}[ampersand replacement=\&,column sep=small,cramped]
    \bullet \& \bullet
    \arrow[""{name=0, anchor=center, inner sep=0}, controls=+(80:0.7) and +(100:0.7), from=1-1, to=1-2]
    \arrow[""{name=1, anchor=center, inner sep=0}, curve={height=0}, from=1-1, to=1-2]
    \arrow[""{name=2, anchor=center, inner sep=0}, controls=+(100:-0.7) and +(80:-0.7),, from=1-1, to=1-2]
    \arrow["\id", shorten <=3pt, shorten >=3pt, Rightarrow, from=2, to=1]
    \arrow["\beta", color=red, shorten <=3pt, shorten >=3pt, Rightarrow, from=1, to=0]
  \end{tikzcd}}}
\newsavebox{\ehleftbot}
\savebox{\ehleftbot}{
  \adjustbox{scale=1}{\begin{tikzcd}[ampersand replacement=\&,column sep=small,cramped]
    \bullet \& \bullet
    \arrow[""{name=0, anchor=center, inner sep=0}, controls=+(80:0.7) and +(100:0.7),, from=1-1, to=1-2]
    \arrow[""{name=1, anchor=center, inner sep=0}, curve={height=0}, from=1-1, to=1-2]
    \arrow[""{name=2, anchor=center, inner sep=0}, controls=+(100:-0.7) and +(80:-0.7),, from=1-1, to=1-2]
    \arrow["\id", shorten <=3pt, shorten >=3pt, Rightarrow, from=2, to=1]
    \arrow["\alpha", color=blue, shorten <=3pt, shorten >=3pt, Rightarrow, from=1, to=0]
  \end{tikzcd}}}
\newsavebox{\ehrightbot}
\savebox{\ehrightbot}{
  \adjustbox{scale=1}{\begin{tikzcd}[ampersand replacement=\&,column sep=small,cramped]
    \bullet \& \bullet
    \arrow[""{name=0, anchor=center, inner sep=0}, controls=+(80:0.7) and +(100:0.7), from=1-1, to=1-2]
    \arrow[""{name=1, anchor=center, inner sep=0}, curve={height=0}, from=1-1, to=1-2]
    \arrow[""{name=2, anchor=center, inner sep=0}, controls=+(100:-0.7) and +(80:-0.7),, from=1-1, to=1-2]
    \arrow["\beta", color=red, shorten <=3pt, shorten >=3pt, Rightarrow, from=2, to=1]
    \arrow["\id", shorten <=3pt, shorten >=3pt, Rightarrow, from=1, to=0]
  \end{tikzcd}}}

\begin{figure}[h]
  \centering

  \[
    \begin{tikzcd}[ampersand replacement=\&,column sep=small]
      \bullet \&\& \bullet \& = \& \bullet \&\&\&\&\& \bullet \& = \& \bullet \&\&\& \bullet \&\&\& \bullet \\
      \\
      \&\&\&\&\&\&\&\&\&\&\&\&\&\& = \\
      \\
      \bullet \&\& \bullet \& = \& \bullet \&\&\&\&\& \bullet \& = \& \bullet \&\&\& \bullet \&\&\& \bullet
      \arrow[""{name=0, anchor=center, inner sep=0}, "\id", curve={height=-24pt}, from=1-1, to=1-3]
      \arrow[""{name=1, anchor=center, inner sep=0}, "\id"', curve={height=24pt}, from=1-1, to=1-3]
      \arrow[""{name=2, anchor=center, inner sep=0}, "\id"{description}, from=1-1, to=1-3]
      \arrow[""{name=3, anchor=center, inner sep=0}, draw=none, controls=+(90:1.8) and +(90:1.8), from=1-5, to=1-10]
      \arrow[""{name=4, anchor=center, inner sep=0}, draw=none, controls=+(90:-1.8) and +(90:-1.8), from=1-5, to=1-10]
      \arrow[""{name=5, anchor=center, inner sep=0}, from=1-5, to=1-10]
      \arrow[""{name=6, anchor=center, inner sep=0}, draw=none, controls=+(90:1.8) and +(90:1.8), from=5-5, to=5-10]
      \arrow[""{name=7, anchor=center, inner sep=0}, draw=none, controls=+(90:-1.8) and +(90:-1.8), from=5-5, to=5-10]
      \arrow[""{name=8, anchor=center, inner sep=0}, from=5-5, to=5-10]
      \arrow[""{name=9, anchor=center, inner sep=0}, "\id", curve={height=-24pt}, from=5-1, to=5-3]
      \arrow[""{name=10, anchor=center, inner sep=0}, "\id"', curve={height=24pt}, from=5-1, to=5-3]
      \arrow[""{name=11, anchor=center, inner sep=0}, "\id"{description}, from=5-1, to=5-3]
      \arrow[""{name=12, anchor=center, inner sep=0}, draw=none, controls=+(80:1.5) and +(100:1.5), from=1-12, to=1-15]
      \arrow[""{name=13, anchor=center, inner sep=0}, draw=none, controls=+(100:-1.5) and +(80:-1.5), from=1-12, to=1-15]
      \arrow[""{name=14, anchor=center, inner sep=0}, draw=none, controls=+(80:1.5) and +(100:1.5), from=1-15, to=1-18]
      \arrow[""{name=15, anchor=center, inner sep=0}, draw=none, controls=+(100:-1.5) and +(80:-1.5), from=1-15, to=1-18]
      \arrow[""{name=16, anchor=center, inner sep=0}, draw=none, controls=+(80:1.5) and +(100:1.5), from=5-12, to=5-15]
      \arrow[""{name=17, anchor=center, inner sep=0}, draw=none, controls=+(100:-1.5) and +(80:-1.5), from=5-12, to=5-15]
      \arrow[""{name=18, anchor=center, inner sep=0}, draw=none, controls=+(80:1.5) and +(100:1.5), from=5-15, to=5-18]
      \arrow[""{name=19, anchor=center, inner sep=0}, draw=none, controls=+(100:-1.5) and +(80:-1.5), from=5-15, to=5-18]
      \arrow["\alpha"', color={rgb,255:red,0;green,24;blue,204}, shorten <=3pt, shorten >=5pt, Rightarrow, from=1, to=2]
      \arrow["\beta"', color={rgb,255:red,204;green,0;blue,14}, shorten <=5pt, shorten >=3pt, Rightarrow, from=2, to=0]
      \arrow["\beta"', color={rgb,255:red,204;green,0;blue,14}, shorten <=3pt, shorten >=5pt, Rightarrow, from=10, to=11]
      \arrow["\alpha"', color={rgb,255:red,0;green,24;blue,204}, shorten <=5pt, shorten >=3pt, Rightarrow, from=11, to=9]
      \arrow["\usebox{\ehalpha}"{description,inner sep = 0,xshift = -1.2pt}, shorten <=3pt, shorten >=3pt, Rightarrow, from=4, to=5]
      \arrow["\usebox{\ehbeta}"{description,inner sep = 0,xshift = -1.2pt}, shorten <=3pt, shorten >=3pt, Rightarrow, from=5, to=3]
      \arrow["\usebox{\ehbeta}"{description,inner sep = 0,xshift = -1.2pt}, shorten <=3pt, shorten >=3pt, Rightarrow, from=7, to=8]
      \arrow["\usebox{\ehalpha}"{description,inner sep = 0,xshift = -1.2pt}, shorten <=3pt, shorten >=3pt, Rightarrow, from=8, to=6]
      \arrow["\usebox{\ehlefttop}"{description,inner sep = 0,xshift = -1.3pt, yshift = 0.2pt}, shorten <=3pt, shorten >=3pt, Rightarrow, from=13, to=12]
      \arrow["\usebox{\ehrighttop}"{description,inner sep = 0,xshift = -1.3pt,yshift = 0.2pt}, shorten <=3pt, shorten >=3pt, Rightarrow, from=15, to=14]
      \arrow["\usebox{\ehleftbot}"{description,inner sep = 0,xshift = -1.3pt}, shorten <=3pt, shorten >=3pt, Rightarrow, from=17, to=16]
      \arrow["\usebox{\ehrightbot}"{description,inner sep = 0,xshift = -1.3pt}, shorten <=3pt, shorten >=3pt, Rightarrow, from=19, to=18]
      \arrow[controls=+(90:1.8) and +(90:1.8), from=1-5, to=1-10]
      \arrow[controls=+(90:-1.8) and +(90:-1.8), from=1-5, to=1-10]
      \arrow[controls=+(90:1.8) and +(90:1.8), from=5-5, to=5-10]
      \arrow[controls=+(90:-1.8) and +(90:-1.8), from=5-5, to=5-10]
      \arrow[controls=+(80:1.5) and +(100:1.5), from=1-12, to=1-15]
      \arrow[controls=+(100:-1.5) and +(80:-1.5), from=1-12, to=1-15]
      \arrow[controls=+(80:1.5) and +(100:1.5), from=1-15, to=1-18]
      \arrow[controls=+(100:-1.5) and +(80:-1.5), from=1-15, to=1-18]
      \arrow[controls=+(80:1.5) and +(100:1.5), from=5-12, to=5-15]
      \arrow[controls=+(100:-1.5) and +(80:-1.5), from=5-12, to=5-15]
      \arrow[controls=+(80:1.5) and +(100:1.5), from=5-15, to=5-18]
      \arrow[controls=+(100:-1.5) and +(80:-1.5), from=5-15, to=5-18]
    \end{tikzcd}
  \]
  \caption{The Eckmann-Hilton argument}
  \label{fig:eh}
\end{figure}

\subsection{Pasting diagrams}
\label{sec:pasting-diagrams}

\todo{Reference batanin and leinster here}

The definition of \(\infty\)-categories given in the previous section is close in spirit to the the ordinary definitions of 1-categories and clearly demonstrates the different families of axioms present. However, we will see in \cref{sec:strictness}, that these sort of definitions do not scale well to our eventual setting of weak higher categories.

There is a special class of (finite) globular sets known as \emph{pasting diagrams}. The elements of the free strict \(\infty\)-category on a globular set \(G\) can instead be represented by a pasting diagram and a map from this pasting diagram into \(G\). To do this, it must be possible to obtain a canonical composite from each pasting diagram.

Informally, we can define a \(n\)-dimensional pasting diagram to be a finite globular set which admits a unique full composite of dimension \(n\), where a full composite of a globular set \(G\) is an element of the free \(\infty\)-category over \(G\) which uses all the maximal elements. This functions as the primary intuition on the role of pasting diagrams.

Before giving a more formal definition of pasting diagrams, we explore some examples and non-examples. A more indepth discussion of pasting diagrams and representations of free strict \(\infty\)-categories using them can be found in \citeauthor{leinster2004higher}~\cite{leinster2004higher}. In contrast to \citeauthor{leinster2004higher}, we consider pasting diagrams as a full subcategory of globular sets, rather than a separate category with a function sending each pasting diagram to a globular set.

The disc contexts introduced in \cref{ex:disc} are all examples of pasting diagrams. The unique ``composite'' of these globular sets is just given by their maximal element, noting that we allow a singular cell in our informal definition of composite. The uniqueness of this is trivial as the only possible operations we could apply are compositions with units, which gives the same cell under the laws of an \(\infty\)-category.

The diagrams used to graphically represent our composition operations (of which we recall three below) are also pasting diagrams.

\[
  \begin{tikzcd}
    x & y & z
    \arrow["f", from=1-1, to=1-2]
    \arrow["g", from=1-2, to=1-3]
  \end{tikzcd}
  \qquad
  \begin{tikzcd}
    x && y
    \arrow[""{name=0, anchor=center, inner sep=0}, "f"', curve={height=24pt}, from=1-1, to=1-3]
    \arrow[""{name=1, anchor=center, inner sep=0}, "h", curve={height=-24pt}, from=1-1, to=1-3]
    \arrow[""{name=2, anchor=center, inner sep=0}, "g"{description}, from=1-1, to=1-3]
    \arrow["\alpha", shorten <=3pt, shorten >=3pt, Rightarrow, from=0, to=2]
    \arrow["\beta", shorten <=3pt, shorten >=3pt, Rightarrow, from=2, to=1]
  \end{tikzcd}
  \qquad
    \begin{tikzcd}
    x & y & z
    \arrow[""{name=0, anchor=center, inner sep=0}, "g", curve={height=-18pt}, from=1-1, to=1-2]
    \arrow[""{name=1, anchor=center, inner sep=0}, "f"', curve={height=18pt}, from=1-1, to=1-2]
    \arrow[""{name=2, anchor=center, inner sep=0}, "i", curve={height=-18pt}, from=1-2, to=1-3]
    \arrow[""{name=3, anchor=center, inner sep=0}, "h"', curve={height=18pt}, from=1-2, to=1-3]
    \arrow["\alpha", shorten <=5pt, shorten >=5pt, Rightarrow, from=1, to=0]
    \arrow["\beta", shorten <=5pt, shorten >=5pt, Rightarrow, from=3, to=2]
  \end{tikzcd}
\]

The composite of these diagrams is just the composite of the two maximal cells with the appropriate codimension.

We can also consider composites which are not binary composites of two cells of equal dimension. For example the following globular set is a pasting diagram:

\[
  \begin{tikzcd}
    x & y & z
    \arrow[""{name=0, anchor=center, inner sep=0}, "g", curve={height=-18pt}, from=1-1, to=1-2]
    \arrow[""{name=1, anchor=center, inner sep=0}, "f"', curve={height=18pt}, from=1-1, to=1-2]
    \arrow["h", from=1-2, to=1-3]
    \arrow["\alpha", shorten <=5pt, shorten >=5pt, Rightarrow, from=1, to=0]
  \end{tikzcd}
\]
with a composite given by \(\alpha \star_0 \id(h)\). This operation is fairly common (in fact we have already seen it in \cref{prop:eh}) and is known as \emph{whiskering}. In this case we would say that the composite is given by the right whiskering of \(\alpha\) with \(h\).

The 1-dimensional pasting diagrams are all given by chains of 1-cells of the form:
\[x_0 \overset{f_0}\to x_1 \overset{f_1}\to x_2 \overset{f_2}\to \cdots \overset{f_n}\to x_{n+1}\]
There are multiple ways to form a composite over these diagrams by repeated binary composition, however these all have the same result due to associativity.

Lastly we look at the diagram, where all the \(0\)-cells and \(1\)-cells are assumed to be distinct:

\[
  \begin{tikzcd}[column sep = large]
    \bullet & \bullet & \bullet
    \arrow[""{name=0, anchor=center, inner sep=0}, curve={height=-30pt}, from=1-1, to=1-2]
    \arrow[""{name=1, anchor=center, inner sep=0}, curve={height=30pt}, from=1-1, to=1-2]
    \arrow[""{name=2, anchor=center, inner sep=0}, from=1-1, to=1-2]
    \arrow[""{name=3, anchor=center, inner sep=0}, curve={height=-30pt}, from=1-2, to=1-3]
    \arrow[""{name=4, anchor=center, inner sep=0}, curve={height=30pt}, from=1-2, to=1-3]
    \arrow[""{name=5, anchor=center, inner sep=0}, from=1-2, to=1-3]
    \arrow["\alpha", shorten <=4pt, shorten >=4pt, Rightarrow, from=1, to=2]
    \arrow["\beta", shorten <=4pt, shorten >=4pt, Rightarrow, from=2, to=0]
    \arrow["\gamma", shorten <=4pt, shorten >=4pt, Rightarrow, from=4, to=5]
    \arrow["\delta", shorten <=4pt, shorten >=4pt, Rightarrow, from=5, to=3]
  \end{tikzcd}
\]

We get a composite given by \((\alpha \star_1 \beta) \star_0 (\gamma \star_1 \delta)\). The uniqueness of this composite is due to the interchange law. \todo{Give some other examples in a figure?}

Non-examples of pasting diagrams roughly fall into two groups: those that do not admit a composite, and those that admit many distinct composites. The following three globular sets fail to admit a composite (the last is drawn in a box to emphasise that \(z\) is part of the same globular set as \(x\), \(y\), \(f\), \(g\), and \(\alpha\)):

\[
  \begin{tikzcd}[column sep=large, row sep = small]
    & y \\
    x \\
    & z
    \arrow["f", pos=0.6, from=2-1, to=1-2]
    \arrow["g"', pos=0.6, from=2-1, to=3-2]
  \end{tikzcd}
  \qquad
  \begin{tikzcd}[column sep=large]
    x & y
    \arrow["f", curve={height=-12pt}, from=1-1, to=1-2]
    \arrow["g"', curve={height=12pt}, from=1-1, to=1-2]
  \end{tikzcd}
  \qquad
  \fbox{\begin{tikzcd}[column sep=scriptsize, ampersand replacement = \&]
    x \&\& y \& z
    \arrow[""{name=0, anchor=center, inner sep=0}, "f", curve={height=-18pt}, from=1-1, to=1-3]
    \arrow[""{name=1, anchor=center, inner sep=0}, "g"', curve={height=18pt}, from=1-1, to=1-3]
    \arrow["\alpha", shorten <=5pt, shorten >=5pt, Rightarrow, from=1, to=0]
  \end{tikzcd}}
\]
The globular set with a single \(0\)-cell \(x\), and a single \(1\)-cell \(f : x \to x\) has too many composites: \(f\) and \(f \star_0 f\) need not be equal in an infinity category.

To describe the free \(\infty\)-category in terms of pasting diagrams we need to be able to extract a composite from a pasting diagram, and construct a pasting diagram from an arbitrary composite. Each pasting diagram having a unique composite solves the former issue.

To be able to construct a pasting diagram from a composite, we wish to equip our set of pasting diagrams itself with the structure of an \(\infty\)-category. We therefore need our pasting diagrams to have a notion of boundary and a notion of composition. A natural candidate for composition is given by colimits, as \(\mathbf{Glob}\) has all colimits due to being a presheaf category, and so it is sufficient for our class of pasting diagrams to be closed under these specific colimits. In fact, it is sufficient to contain a class of colimits known as \emph{globular sums}.

\begin{definition}
  A globular category is a category \(\mathcal{C}\), equipped with a functor \(\mathbf{G} \to \mathcal{C}\), specifying certain objects as discs in the category. In a globular category, we write \(D^n\) for the image of \(n\) under this functor. A \emph{globular sum} is a colimit of a diagram of the form:
  \[
    \begin{tikzcd}[column sep = tiny, row sep = tiny]
      {D^{i_0}} && {D^{i_1}} && {D^{i_2}} && {D^{i_n}} && {D^{i_{n+1}}} \\
      &&&&& \cdots \\
      & {D^{j_0}} && {D^{j_1}} &&&& {D^{j_n}}
      \arrow["{f_0}", from=3-2, to=1-1]
      \arrow["{g_0}"', from=3-2, to=1-3]
      \arrow["{f_n}", from=3-8, to=1-7]
      \arrow["{g_n}"', from=3-8, to=1-9]
      \arrow["{f_1}", from=3-4, to=1-3]
      \arrow["{g_1}"', from=3-4, to=1-5]
    \end{tikzcd}
  \]
  Where all morphisms \(f_i, g_i\) are in the image of the functor \(\mathbf{G} \to \mathcal{C}\).
\end{definition}

We can finally give our first definition of a pasting diagram.

\todo[inline]{Reference Ara?}
\begin{definition}
  The category \(\mathbf{Glob}\) is a globular category with functor \(\mathbf{G} \to \mathbf{Glob}\) given by the Yoneda embedding. The category of \emph{pasting diagrams}, \(\mathbf{Pd}\), is the full subcategory containing the globular sets which are globular sums.
  The boundary of an \((n+1)\)-dimensional pasting diagram is given by replacing each instance of \(D^{n+1}\) by \(D^n\) in its globular sum representation. There are two canonical maps including the boundary into the original pasting diagram, whose images give the source and target of the pasting diagram.
\end{definition}

We finish this section with one larger example.

\begin{example}
  The following depicts a \(2\)-dimensional pasting diagram.
  \[
    \begin{tikzcd}
      x & y & z & w
      \arrow[""{name=0, anchor=center, inner sep=0}, "g", curve={height=-18pt}, from=1-1, to=1-2]
      \arrow[""{name=1, anchor=center, inner sep=0}, "f"', curve={height=18pt}, from=1-1, to=1-2]
      \arrow["h"', from=1-2, to=1-3]
      \arrow[""{name=2, anchor=center, inner sep=0}, "k", curve={height=-24pt}, from=1-3, to=1-4]
      \arrow[""{name=3, anchor=center, inner sep=0}, "i"', curve={height=24pt}, from=1-3, to=1-4]
      \arrow[""{name=4, anchor=center, inner sep=0}, "j"{description}, from=1-3, to=1-4]
      \arrow["\alpha", shorten <=5pt, shorten >=5pt, Rightarrow, from=1, to=0]
      \arrow["\beta", shorten <=3pt, shorten >=3pt, Rightarrow, from=3, to=4]
      \arrow["\gamma", shorten <=3pt, shorten >=3pt, Rightarrow, from=4, to=2]
    \end{tikzcd}
  \]
  which has the following globular sum decomposition:
  % https://q.uiver.app/#q=WzAsMTMsWzAsMCwieCJdLFsyLDAsInkiXSxbOCwwLCJ6Il0sWzEwLDAsInciXSxbMywxLCJ5Il0sWzQsMCwieSJdLFs2LDAsInoiXSxbNywxLCJ6Il0sWzksMF0sWzEwLDEsInoiXSxbMTIsMSwidyJdLFsxMiwwLCJ6Il0sWzE0LDAsInciXSxbMCwxLCJnIiwwLHsiY3VydmUiOi0zfV0sWzAsMSwiZiIsMix7ImN1cnZlIjozfV0sWzIsMywiaSIsMix7ImN1cnZlIjo0fV0sWzIsMywiaiIsMV0sWzUsNiwiaCIsMl0sWzQsMSwiIiwyLHsic3R5bGUiOnsiYm9keSI6eyJuYW1lIjoiZGFzaGVkIn19fV0sWzQsNSwiIiwxLHsic3R5bGUiOnsiYm9keSI6eyJuYW1lIjoiZGFzaGVkIn19fV0sWzksMTAsImoiLDFdLFsxMSwxMiwiaiIsMV0sWzExLDEyLCJrIiwxLHsiY3VydmUiOi00fV0sWzcsNiwiIiwwLHsic3R5bGUiOnsiYm9keSI6eyJuYW1lIjoiZGFzaGVkIn19fV0sWzcsMiwiIiwwLHsic3R5bGUiOnsiYm9keSI6eyJuYW1lIjoiZGFzaGVkIn19fV0sWzE0LDEzLCJcXGFscGhhIiwwLHsic2hvcnRlbiI6eyJzb3VyY2UiOjIwLCJ0YXJnZXQiOjIwfX1dLFsxNSwxNiwiXFxiZXRhIiwwLHsic2hvcnRlbiI6eyJzb3VyY2UiOjIwLCJ0YXJnZXQiOjIwfX1dLFsyMSwyMiwiXFxnYW1tYSIsMCx7InNob3J0ZW4iOnsic291cmNlIjoyMCwidGFyZ2V0IjoyMH19XSxbMjAsMywiIiwxLHsic2hvcnRlbiI6eyJzb3VyY2UiOjIwfSwibGV2ZWwiOjEsInN0eWxlIjp7ImJvZHkiOnsibmFtZSI6ImRhc2hlZCJ9fX1dLFsyMCwxMSwiIiwxLHsic2hvcnRlbiI6eyJzb3VyY2UiOjIwfSwibGV2ZWwiOjEsInN0eWxlIjp7ImJvZHkiOnsibmFtZSI6ImRhc2hlZCJ9fX1dXQ==
  \[
    \begin{tikzcd}[column sep=tiny, row sep = small]
      x && y && y && z && z & {} & w && z && w \\
      &&& y &&&& z &&& z && w
      \arrow[""{name=0, anchor=center, inner sep=0}, "g", curve={height=-18pt}, from=1-1, to=1-3]
      \arrow[""{name=1, anchor=center, inner sep=0}, "f"', curve={height=18pt}, from=1-1, to=1-3]
      \arrow[""{name=2, anchor=center, inner sep=0}, "i"', curve={height=24pt}, from=1-9, to=1-11]
      \arrow[""{name=3, anchor=center, inner sep=0}, "j"{description}, from=1-9, to=1-11]
      \arrow["h"', from=1-5, to=1-7]
      \arrow[dashed, from=2-4, to=1-3]
      \arrow[dashed, from=2-4, to=1-5]
      \arrow[""{name=4, anchor=center, inner sep=0}, "j"{description}, from=2-11, to=2-13]
      \arrow[""{name=5, anchor=center, inner sep=0}, "j"{description}, from=1-13, to=1-15]
      \arrow[""{name=6, anchor=center, inner sep=0}, "k"{description}, curve={height=-24pt}, from=1-13, to=1-15]
      \arrow[dashed, from=2-8, to=1-7]
      \arrow[dashed, from=2-8, to=1-9]
      \arrow["\alpha", shorten <=5pt, shorten >=5pt, Rightarrow, from=1, to=0]
      \arrow["\beta", shorten <=3pt, shorten >=3pt, Rightarrow, from=2, to=3]
      \arrow["\gamma", shorten <=3pt, shorten >=3pt, Rightarrow, from=5, to=6]
      \arrow[shorten <=6pt, dashed, from=4, to=1-11]
      \arrow[shorten <=6pt, dashed, from=4, to=1-13]
    \end{tikzcd}
  \]
  The source and target of the diagram are given by:
  \[
    \begin{tikzcd}
      x & y & z & w
      \arrow["g", curve={height=-18pt}, from=1-1, to=1-2]
      \arrow["h", from=1-2, to=1-3]
      \arrow["k", curve={height=-24pt}, from=1-3, to=1-4]
    \end{tikzcd}
    \qquad\text{and}\qquad
    \begin{tikzcd}
      x & y & z & w
      \arrow["f"', curve={height=18pt}, from=1-1, to=1-2]
      \arrow["h", from=1-2, to=1-3]
      \arrow["i"', curve={height=24pt}, from=1-3, to=1-4]
    \end{tikzcd}
  \]
  which are isomorphic pasting diagrams.
\end{example}
\todo{This diagram looks awful}

\subsection{Strictness}
\label{sec:strictness}

The \(\infty\)-categories we have defined so far have all been strict \(\infty\)-categories, meaning that the laws are required to hold up to equality. In ordinary \(1\)-category theory, isomorphism is usually preferred over equality for comparing objects. Similarly, when we have access to higher dimensional arrows, it follows that that we can also consider isomorphisms between morphisms, and therefore consider laws such as associativity up to isomorphism instead of equality.

Topological spaces provide one of the primary examples for where it is useful to consider weak laws. Given a topological space \(X\), we can define a globular set of paths and homotopies. Let the \(0\)-cells be given by points \(x\) of the topological space, let morphisms from \(x\) to \(y\) be given as paths \(I \to X\) (where \(I\) is the topological interval \([0,1]\)) which send \(0\) to \(x\) and \(1\) to \(y\), and let higher cells be given by homotopies. The natural composition of two paths \(p\) and \(q\) is the following path:
\[
  (p \star q)(i) =
  \begin{cases*}
    p(2i)&when \(i < 0.5\)\\
    q(2i-1)&when \(i \geq 0.5\)
  \end{cases*}
\]
which effectively lines up the paths end to end. Given \(3\) paths \(p\), \(q\), and \(r\), the compositions \((p \star q) \star r\) and \(p \star (q \star r)\) are not identical but are equal up to homotopy, meaning the two compositions are isomorphic. Therefore in this case the composition \(p \star q\) does not form a strict \(\infty\)-category structure, but rather a weak structure.

We start our exploration of weak higher categories by considering the lower dimension case of bicategories (weak \(2\)-categories). Here, interchange must still be given by a strict equality, as there are no non-trivial \(3\)-cells in a \(2\)-category. However, associativity and unitality can be given by isomorphisms known as associators and unitors:
\begin{alignat*}{2}
  &\alpha_{f,g,h} &&: (f \star_0 g) \star_0 h \to f \star_0 (g \star_0 h)\\
  &\lambda_f &&: \id(x) \star_0 f \to f\\
  &\rho_f &&: f \star_0 \id(y) \to f
\end{alignat*}
for \(f : x \to y\), \(g : y \to z\), and \(h : z \to w\).

\begin{example}
  \label{ex:spans}
  All strict 2-categories are also bicategories. The bicategory of spans is an example of a bicategory which is not strict. Starting with a category \(\mathcal{C}\) equipped with chosen pullbacks, we define the bicategory of spans over \(\mathcal{C}\) to be:
  \begin{itemize}
  \item Objects are the same as \(\mathcal{C}\)
  \item Morphisms \(A\) to \(B\) are spans \(A \leftarrow C \to B\).
  \item A 2-morphism from \(A \leftarrow C \to B\) to \(A \leftarrow C' \to B\) is a morphism \(C \to C'\) such that the following diagram commutes:
    \[
      \begin{tikzcd}[row sep = small]
	& C \\
	A && B \\
	& {C'}
	\arrow[from=1-2, to=3-2]
	\arrow[from=3-2, to=2-1]
	\arrow[from=1-2, to=2-1]
	\arrow[from=1-2, to=2-3]
	\arrow[from=3-2, to=2-3]
      \end{tikzcd}
    \]
  \item Compositions and identities of 2-morphisms is given by composition and identities of the underlying morphisms in \(\mathcal{C}\).
  \item The identity on an object \(A\) is the span \(A \leftarrow A \to A\).
  \item Given spans \(A \leftarrow D \to B\) and \(B \leftarrow E \to C\), their composite is given by the pullback:
    \[
      \begin{tikzcd}[row sep=small]
	&& {D \times_B E} \\
	& D && E \\
	A && B && C
	\arrow[from=2-2, to=3-1]
	\arrow[from=2-2, to=3-3]
	\arrow[from=2-4, to=3-3]
	\arrow[from=2-4, to=3-5]
	\arrow[from=1-3, to=2-2]
	\arrow[from=1-3, to=2-4]
	\arrow["\lrcorner"{anchor=center, pos=0.125, rotate=-45}, draw=none, from=1-3, to=3-3]
      \end{tikzcd}
    \]
  \item Associators and unitors are given by the universal property of the pullback.
  \end{itemize}
\end{example}

In general, there could be many possible isomorphisms between \((f \star g) \star h\) and \(f \star (g \star h)\), and we require that the chosen morphisms satisfy certain compatibility properties. The first is that each of the associator, left unitor, and right unitor should be a natural isomorphism. The second is a property known as \emph{coherence}, saying that any two parallel morphisms built purely from naturality moves, associators, and unitors must be equal.

For bicategories it is sufficient to give two coherence laws: the triangle equality and pentagon equality. The triangle equality identifies two ways of cancelling the identity in the composite \(f \star \id \star g\), giving a compatibility between the left and right unitors. It is given by the following commutative diagram:

% https://q.uiver.app/#q=WzAsMyxbMCwwLCIoZiBcXHN0YXIgXFxpZCkgXFxzdGFyIGciXSxbMiwwLCJmIFxcc3RhciAoXFxpZCBcXHN0YXIgZykiXSxbMSwxLCJmIFxcc3RhciBnIl0sWzAsMSwiXFxhbHBoYV97ZixcXGlkLGd9Il0sWzAsMiwiXFxyaG9fZiBcXHN0YXJfMCBcXGlkKGcpIiwyXSxbMSwyLCJcXGlkKGYpXFxzdGFyXzBcXGxhbWJkYV9nIl1d
\[
  \begin{tikzcd}
    {(f \star \id) \star g} && {f \star (\id \star g)} \\
    & {f \star g}
    \arrow["{\alpha_{f,\id,g}}", from=1-1, to=1-3]
    \arrow["{\rho_f \star_0 \id(g)}"', from=1-1, to=2-2]
    \arrow["{\id(f)\star_0\lambda_g}", from=1-3, to=2-2]
  \end{tikzcd}
\]

The pentagon equation identifies two ways of associating \(((f \star g) \star h) \star k\) to \(f \star (g \star (h \star k))\). It is given by the diagram below:
% https://q.uiver.app/#q=WzAsNSxbMSwzLCIoZiBcXHN0YXIgKGcgXFxzdGFyIGgpKSBcXHN0YXIgayJdLFswLDEsIigoZiBcXHN0YXIgZykgXFxzdGFyIGgpIFxcc3RhciBrIl0sWzIsMCwiKGYgXFxzdGFyIGcpIFxcc3RhciAoaCBcXHN0YXIgaykiXSxbNCwxLCJmIFxcc3RhciAoZyBcXHN0YXIgKGggXFxzdGFyIGspKSJdLFszLDMsImYgXFxzdGFyICgoZyBcXHN0YXIgaCkgXFxzdGFyIGspIl0sWzEsMiwiXFxhbHBoYV97ZiBcXHN0YXIgZyxoLGt9Il0sWzIsMywiXFxhbHBoYV97ZixnLGhcXHN0YXIga30iXSxbMSwwLCJcXGFscGhhX3tmLGcsaH0gXFxzdGFyXzAgXFxpZChrKSIsMl0sWzAsNCwiXFxhbHBoYV97ZixnXFxzdGFyIGgsa30iLDJdLFs0LDMsIlxcaWQoZilcXHN0YXJfMCBcXGFscGhhX3tnLGgsa30iLDJdXQ==
\[
  \begin{tikzcd}[column sep = -1.5em]
    && {(f \star g) \star (h \star k)} \\
    {((f \star g) \star h) \star k} &&&& {f \star (g \star (h \star k))} \\
    \\
    & {(f \star (g \star h)) \star k} && {f \star ((g \star h) \star k)}
    \arrow["{\alpha_{f \star g,h,k}}", from=2-1, to=1-3]
    \arrow["{\alpha_{f,g,h\star k}}", from=1-3, to=2-5]
    \arrow["{\alpha_{f,g,h} \star_0 \id(k)}"', from=2-1, to=4-2]
    \arrow["{\alpha_{f,g\star h,k}}"', from=4-2, to=4-4]
    \arrow["{\id(f)\star_0 \alpha_{g,h,k}}"', from=4-4, to=2-5]
  \end{tikzcd}
\]

Surprisingly, these two equations are enough to give full coherence. For the example of spans from \cref{ex:spans}, these two equations follow from the uniqueness of the universal morphism.

To move from weak \(2\)-categories to weak \(3\)-categories, new coherence cells for interchangers are added to replace the interchanger equalities, and new equalities must be added to specify the interaction between the interchangers and other coherence morphisms. Furthermore, the triangle and pentagon equation from \(2\)-categories will become an isomorphisms in a weak \(3\)-category, causing more coherence equations to be added.

As we move up in dimension, the number of coherence morphisms and equalities required increases exponentially.

\todo[inline]{Exponential blowup in number of coherence moves needed}

Because of this complexity, we look for more uniform ways to represent the operations and axioms of an \(\infty\)-category. In this thesis, we will work with the type theory \Catt, which is based on a definition of \(\infty\)-categories due to \citeauthor{maltsiniotis2010grothendieck}~\cite{maltsiniotis2010grothendieck}, which is itself based on a definition of \(\infty\)-groupoid by \citeauthor{PursuingStacks}~\cite{PursuingStacks}. We will sketch the ideas behind these definitions here, and give a definition of \Catt in \cref{sec:type-theory-catt}.

\todo[inline]{Grothendieck definition}

\todo[inline]{Implies invertibility}

\todo[inline]{Maltsiniotis definition}

\todo[inline]{Eckmann-Hilton in weak setting}

\todo[inline]{Strictification for bicategories}

\todo[inline]{Braiding and counter example for 3-dimension strictification}



\todo[inline]{Conjectures}

\section{Type theory preliminaries}
\label{sec:type-theory}

\todo[inline]{Background on type theory}

\todo[inline]{Dependent type theory}

\todo[inline]{MLTT/HoTT}

\todo[inline]{Properties of type theories}

\section{The type theory \Catt}
\label{sec:type-theory-catt}

\todo[inline]{High level presentation of Catt}

\todo[inline]{We give a more formal definition in the next section}

\todo[inline]{Talk about models of Catt here}

\chapter{A general presentation of \Catt}
\label{cha:gener-pres-catt}

\todo[inline]{Should be able take this mainly from cattsua paper}

\section{\Catt with equality}
\label{sec:catt-with-equality}

\section{Properties of \Catt with equality}
\label{sec:properties-catt-with}

\section{Support conditions}
\label{sec:support-conditions}

\todo[inline]{Ideas about invertibility here?}

\chapter{Constructions in \Catt}
\label{sec:operations-catt}

\section{Basic constructions}
\label{sec:basic-constructions}

\section{Structured terms}
\label{sec:structured-terms}

\section{Pruning}
\label{sec:pruning}

\section{Insertion}
\label{sec:insertion}

\chapter{Semistrict versions of \Catt}
\label{cha:cattsu}

\section{\Cattsu}
\label{sec:cattsu}

\section{\Cattsua}
\label{sec:cattsua}

\section{Rehydration}
\label{sec:rehydration}

\todo[inline]{Explain original motivation}

\todo[inline]{Explain connection to equivalence}

\todo[inline]{core problem with rehydration}

\subsection{Rehydration by dimension}
\label{sec:rehydr-dimens}

\subsection{Rehydration by cylinders}
\label{sec:rehydr-cylind}


















%%%%%%%%%%%%%%%%%%%%%%%%%%%%%%%%%%%%%%%%%%%%%%%%%%%%%%%%%%%%%%%%%%%%%%%%%%%%%%%%
%% References:
%%

\printbibliography

\end{document}
