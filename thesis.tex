\documentclass{cam-thesis}

\usepackage{packages}
\usepackage{macros}

\title{A type-theoretic approach to semistrict higher categories}

%% The full name of the author (e.g.: James Smith):
\author{Alex Rice}

%% College affiliation:
\college{Darwin College}

%% College shield:
\collegeshield{CollegeShields/Darwin}

%% Submission date [optional]:
\submissiondate{TODO}

%% Declaration date:
\date{TODO}

%% PDF meta-info:
\subjectline{Computer Science}
\keywords{category theory, higher category theory, type theory}



%%%%%%%%%%%%%%%%%%%%%%%%%%%%%%%%%%%%%%%%%%%%%%%%%%%%%%%%%%%%%%%%%%%%%%%%%%%%%%%%
%% Abstract:
%%
\abstract{%
  Abstract to go here...
}


%%%%%%%%%%%%%%%%%%%%%%%%%%%%%%%%%%%%%%%%%%%%%%%%%%%%%%%%%%%%%%%%%%%%%%%%%%%%%%%%
%% Acknowledgements:
%%
\acknowledgements{%
  My acknowledgements ...
}


%%%%%%%%%%%%%%%%%%%%%%%%%%%%%%%%%%%%%%%%%%%%%%%%%%%%%%%%%%%%%%%%%%%%%%%%%%%%%%%%
%% Contents:
%%
\begin{document}

%%%%%%%%%%%%%%%%%%%%%%%%%%%%%%%%%%%%%%%%%%%%%%%%%%%%%%%%%%%%%%%%%%%%%%%%%%%%%%%%
%% Title page, abstract, declaration etc.:
%% -    the title page (is automatically omitted in the technical report mode).
\frontmatter{}



%%%%%%%%%%%%%%%%%%%%%%%%%%%%%%%%%%%%%%%%%%%%%%%%%%%%%%%%%%%%%%%%%%%%%%%%%%%%%%%%
%% Thesis body:
%%
\chapter{Introduction}
Things to go in introduction:
\begin{itemize}
\item Motivation for higher categories
\item Motivation for semistrictness
\item Introduce 3 classes of coherences
\item Ideas of semistrictness
\item Link to graphical ideas
\item Story behind Catt - HoTT - Brunerei
\item Link to Grothedieck higher cats
\end{itemize}

\chapter{Background}
\label{sec:background}

We begin with an overview of the important concepts required for the rest of the thesis. Throughout, we will assume knowledge of various basic concepts from computer science, as well as a basic knowledge of category theory (including functor categories, presheafs, and (co)limits). The following sections introduce a form of globular higher categories, present the necessary type theory prerequisites, and give a definition of the type theory \Catt, close to the original definition.

This section additionally serves as a place to introduce the various syntax and notation which will be used throughout the rest of the thesis.

\section{Higher categories}
\label{sec:higher-categories}

A higher category is a generalisation of the ordinary notion of a category to allow higher dimensional structure. This manifests in the form of allowing arrows or morphisms to have their source or target be another morphism instead of an object. More precisely, higher categories are equipped with the notion of an \(n\)-cell, where an \((n+1)\)-cell has source and target \(n\)-cells, and \(0\)-cells play the role of objects in an ordinary category.

The role of objects is played by \(0\)-cells, with \(1\)-cells as the morphisms between these objects. For \(0\)-cells \(x\) and \(y\), a \(1\)-cell \(f\) with source \(x\) and target \(y\) will be drawn as:
\[
  \begin{tikzcd}
    x & y
    \arrow["f", from=1-1, to=1-2]
  \end{tikzcd}
\]
or may be written as \(f: x \to y\).\todo{should arrowheads be smaller?} Two cells are \emph{parallel} if they have the same source and target. Between any two parallel \(n\)-cells \(f\) and \(g\), we have a set of \((n+1)\)-cells between them. A \(2\)-cell \(\alpha : f \to g\) may be drawn as:
\[
  \begin{tikzcd}
    x & y
    \arrow[""{name=0, anchor=center, inner sep=0}, "f", curve={height=-12pt}, from=1-1, to=1-2]
    \arrow[""{name=1, anchor=center, inner sep=0}, "g"', curve={height=12pt}, from=1-1, to=1-2]
    \arrow["\alpha", shorten <=3pt, shorten >=3pt, Rightarrow, nfold, from=1, to=0]
  \end{tikzcd}
\]
and a \(3\)-cell \(\gamma\) between parallel \(2\)-cells \(\alpha\) and \(\beta\) could be could be drawn as:
\[
  \begin{tikzcd}
    x && y
    \arrow[""{name=0, anchor=center, inner sep=0}, "f", curve={height=-15pt}, from=1-1, to=1-3]
    \arrow[""{name=1, anchor=center, inner sep=0}, "g"', curve={height=15pt}, from=1-1, to=1-3]
    \arrow[""{name=2, anchor=center, inner sep=0}, "\alpha", shift left=4, shorten <=3pt, shorten >=3pt, Rightarrow, nfold, from=1, to=0]
    \arrow[""{name=3, anchor=center, inner sep=0}, "\beta"', shift right=4, shorten <=3pt, shorten >=3pt, Rightarrow, nfold, from=1, to=0]
    \arrow["\gamma", shorten <=4pt, shorten >=4pt, Rightarrow, nfold=3, from=2, to=3]
  \end{tikzcd}
\]

Just as in ordinary \(1\)-category theory, we expect to be able to compose morphisms. For \(1\)-cells, nothing has changed, given \(1\)-cells \(f: x \to y\) and \(g : y \to z\) we form the composition \(f \star g\):
\[
  \begin{tikzcd}
    x & y & z
    \arrow[from=1-1, to=1-2, "f"]
    \arrow[from=1-2, to=1-3, "g"]
  \end{tikzcd}
\]
which has source \(x\) and target \(z\). We pause here to note that composition will be given in ``diagrammatic order'' throughout the whole thesis, which is the opposite of the order of function composition but the same as the order of the arrows if drawn head to tail. This is chosen as it will be common for us to draw higher dimensional arrows in a diagram, and rare for us to consider categories where the higher arrows are given by functions. In an attempt to avoid confusion, we use a star (\(\star\)) to represent composition of arrows or cells in a higher category, and will use a circle (\(\circ\)) only for function composition.

In two dimensions, there is no longer a singular composition operation. For \(2\)-cells \(\alpha : f \to g\) and \(\beta : g \to h\), the composite \(\alpha \star_1 \beta\) can be formed as before:
% https://q.uiver.app/#q=WzAsMixbMCwwLCJcXGJ1bGxldCJdLFsyLDAsIlxcYnVsbGV0Il0sWzAsMSwiZiIsMCx7ImN1cnZlIjotNH1dLFswLDEsImgiLDIseyJjdXJ2ZSI6NH1dLFswLDEsImciLDFdLFsyLDQsIlxcYWxwaGEiLDAseyJzaG9ydGVuIjp7InNvdXJjZSI6MjAsInRhcmdldCI6MjB9fV0sWzQsMywiXFxiZXRhIiwwLHsic2hvcnRlbiI6eyJzb3VyY2UiOjIwLCJ0YXJnZXQiOjIwfX1dXQ==
\[
  \begin{tikzcd}
    x && y
    \arrow[""{name=0, anchor=center, inner sep=0}, "f"', curve={height=24pt}, from=1-1, to=1-3]
    \arrow[""{name=1, anchor=center, inner sep=0}, "h", curve={height=-24pt}, from=1-1, to=1-3]
    \arrow[""{name=2, anchor=center, inner sep=0}, "g"{description}, from=1-1, to=1-3]
    \arrow["\alpha", shorten <=3pt, shorten >=3pt, Rightarrow, nfold, from=0, to=2]
    \arrow["\beta", shorten <=3pt, shorten >=3pt, Rightarrow, nfold, from=2, to=1]
  \end{tikzcd}
\]

We refer to this composition as \emph{vertical composition}. We can also compose cells \(\alpha\) and \(\beta\) in the following way:

% https://q.uiver.app/#q=WzAsMyxbMCwwLCJ4Il0sWzEsMCwieSJdLFsyLDAsInoiXSxbMCwxLCIiLDAseyJjdXJ2ZSI6LTN9XSxbMCwxLCIiLDIseyJjdXJ2ZSI6M31dLFsxLDIsIiIsMix7ImN1cnZlIjotM31dLFsxLDIsIiIsMix7ImN1cnZlIjozfV0sWzMsNCwiXFxhbHBoYSIsMCx7InNob3J0ZW4iOnsic291cmNlIjoyMCwidGFyZ2V0IjoyMH19XSxbNSw2LCJcXGJldGEiLDAseyJzaG9ydGVuIjp7InNvdXJjZSI6MjAsInRhcmdldCI6MjB9fV1d
\[
  \begin{tikzcd}
    x & y & z
    \arrow[""{name=0, anchor=center, inner sep=0}, "g", curve={height=-18pt}, from=1-1, to=1-2]
    \arrow[""{name=1, anchor=center, inner sep=0}, "f"', curve={height=18pt}, from=1-1, to=1-2]
    \arrow[""{name=2, anchor=center, inner sep=0}, "i", curve={height=-18pt}, from=1-2, to=1-3]
    \arrow[""{name=3, anchor=center, inner sep=0}, "h"', curve={height=18pt}, from=1-2, to=1-3]
    \arrow["\alpha", shorten <=5pt, shorten >=5pt, Rightarrow, nfold, from=1, to=0]
    \arrow["\beta", shorten <=5pt, shorten >=5pt, Rightarrow, nfold, from=3, to=2]
  \end{tikzcd}
\]

This composition is called the \emph{horizontal composition}, and is written \(\alpha \star_0 \beta\). The subscript refers to the dimension of the shared boundary in the composition, with the \(1\)-cell \(g\) being the shared boundary in the vertical composition example and the \(0\)-cell \(y\) being the shared boundary in the horizontal composition example. The dimension of this shared boundary is the \emph{codimension} of the composition.

This pattern continues with \(3\)-cells, which can be composed at codimension \(0\), \(1\), or \(2\), as below:

% https://q.uiver.app/#q=WzAsNyxbMiwwLCJcXGJ1bGxldCJdLFswLDAsIlxcYnVsbGV0Il0sWzMsMCwiXFxidWxsZXQiXSxbNSwwLCJcXGJ1bGxldCJdLFs2LDAsIlxcYnVsbGV0Il0sWzcsMCwiXFxidWxsZXQiXSxbOCwwLCJcXGJ1bGxldCJdLFsxLDAsIiIsMCx7ImN1cnZlIjotM31dLFsxLDAsIiIsMix7ImN1cnZlIjozfV0sWzIsMywiIiwwLHsiY3VydmUiOi00fV0sWzIsMywiIiwyLHsiY3VydmUiOjR9XSxbMiwzXSxbNCw1LCIiLDAseyJjdXJ2ZSI6LTN9XSxbNCw1LCIiLDIseyJjdXJ2ZSI6M31dLFs1LDYsIiIsMix7ImN1cnZlIjotM31dLFs1LDYsIiIsMix7ImN1cnZlIjozfV0sWzgsNywiIiwyLHsib2Zmc2V0IjotNSwic2hvcnRlbiI6eyJzb3VyY2UiOjIwLCJ0YXJnZXQiOjIwfX1dLFs4LDcsIiIsMCx7Im9mZnNldCI6NSwic2hvcnRlbiI6eyJzb3VyY2UiOjIwLCJ0YXJnZXQiOjIwfX1dLFs4LDcsIiIsMix7InNob3J0ZW4iOnsic291cmNlIjoyMCwidGFyZ2V0IjoyMH19XSxbMTAsMTEsIiIsMix7Im9mZnNldCI6LTQsInNob3J0ZW4iOnsic291cmNlIjoyMCwidGFyZ2V0IjoyMH19XSxbMTAsMTEsIiIsMCx7Im9mZnNldCI6NCwic2hvcnRlbiI6eyJzb3VyY2UiOjIwLCJ0YXJnZXQiOjIwfX1dLFsxMSw5LCIiLDEseyJvZmZzZXQiOi00LCJzaG9ydGVuIjp7InNvdXJjZSI6MjAsInRhcmdldCI6MjB9fV0sWzExLDksIiIsMSx7Im9mZnNldCI6NCwic2hvcnRlbiI6eyJzb3VyY2UiOjIwLCJ0YXJnZXQiOjIwfX1dLFsxMywxMiwiIiwyLHsib2Zmc2V0IjotMywic2hvcnRlbiI6eyJzb3VyY2UiOjIwLCJ0YXJnZXQiOjIwfX1dLFsxMywxMiwiIiwwLHsib2Zmc2V0IjozLCJzaG9ydGVuIjp7InNvdXJjZSI6MjAsInRhcmdldCI6MjB9fV0sWzE1LDE0LCIiLDIseyJvZmZzZXQiOi0zLCJzaG9ydGVuIjp7InNvdXJjZSI6MjAsInRhcmdldCI6MjB9fV0sWzE1LDE0LCIiLDAseyJvZmZzZXQiOjMsInNob3J0ZW4iOnsic291cmNlIjoyMCwidGFyZ2V0IjoyMH19XSxbMTYsMTgsIlxcZ2FtbWEiLDAseyJzaG9ydGVuIjp7InNvdXJjZSI6MjAsInRhcmdldCI6MjB9fV0sWzE4LDE3LCJcXGRlbHRhIiwwLHsic2hvcnRlbiI6eyJzb3VyY2UiOjIwLCJ0YXJnZXQiOjIwfX1dLFsyMSwyMiwiXFxnYW1tYSIsMCx7InNob3J0ZW4iOnsic291cmNlIjoyMCwidGFyZ2V0IjoyMH19XSxbMTksMjAsIlxcZGVsdGEiLDAseyJzaG9ydGVuIjp7InNvdXJjZSI6MjAsInRhcmdldCI6MjB9fV0sWzIzLDI0LCJcXGdhbW1hIiwwLHsic2hvcnRlbiI6eyJzb3VyY2UiOjIwLCJ0YXJnZXQiOjIwfX1dLFsyNSwyNiwiXFxkZWx0YSIsMCx7InNob3J0ZW4iOnsic291cmNlIjoyMCwidGFyZ2V0IjoyMH19XV0=
\[
  \begin{tikzcd}
    \bullet && \bullet & \bullet && \bullet & \bullet & \bullet & \bullet
    \arrow[""{name=0, anchor=center, inner sep=0}, curve={height=-18pt}, from=1-1, to=1-3]
    \arrow[""{name=1, anchor=center, inner sep=0}, curve={height=18pt}, from=1-1, to=1-3]
    \arrow[""{name=2, anchor=center, inner sep=0}, curve={height=-24pt}, from=1-4, to=1-6]
    \arrow[""{name=3, anchor=center, inner sep=0}, curve={height=24pt}, from=1-4, to=1-6]
    \arrow[""{name=4, anchor=center, inner sep=0}, from=1-4, to=1-6]
    \arrow[""{name=5, anchor=center, inner sep=0}, curve={height=-18pt}, from=1-7, to=1-8]
    \arrow[""{name=6, anchor=center, inner sep=0}, curve={height=18pt}, from=1-7, to=1-8]
    \arrow[""{name=7, anchor=center, inner sep=0}, curve={height=-18pt}, from=1-8, to=1-9]
    \arrow[""{name=8, anchor=center, inner sep=0}, curve={height=18pt}, from=1-8, to=1-9]
    \arrow[""{name=9, anchor=center, inner sep=0}, shift left=5, shorten <=5pt, shorten >=5pt, Rightarrow, nfold, from=1, to=0]
    \arrow[""{name=10, anchor=center, inner sep=0}, shift right=5, shorten <=5pt, shorten >=5pt, Rightarrow, nfold, from=1, to=0]
    \arrow[""{name=11, anchor=center, inner sep=0}, shorten <=5pt, shorten >=5pt, Rightarrow, nfold, from=1, to=0]
    \arrow[""{name=12, anchor=center, inner sep=0}, shift left=4, shorten <=3pt, shorten >=3pt, Rightarrow, nfold, from=3, to=4]
    \arrow[""{name=13, anchor=center, inner sep=0}, shift right=4, shorten <=3pt, shorten >=3pt, Rightarrow, nfold, from=3, to=4]
    \arrow[""{name=14, anchor=center, inner sep=0}, shift left=4, shorten <=3pt, shorten >=3pt, Rightarrow, nfold, from=4, to=2]
    \arrow[""{name=15, anchor=center, inner sep=0}, shift right=4, shorten <=3pt, shorten >=3pt, Rightarrow, nfold, from=4, to=2]
    \arrow[""{name=16, anchor=center, inner sep=0}, shift left=3, shorten <=5pt, shorten >=5pt, Rightarrow, nfold, from=6, to=5]
    \arrow[""{name=17, anchor=center, inner sep=0}, shift right=3, shorten <=5pt, shorten >=5pt, Rightarrow, nfold, from=6, to=5]
    \arrow[""{name=18, anchor=center, inner sep=0}, shift left=3, shorten <=5pt, shorten >=5pt, Rightarrow, nfold, from=8, to=7]
    \arrow[""{name=19, anchor=center, inner sep=0}, shift right=3, shorten <=5pt, shorten >=5pt, Rightarrow, nfold, from=8, to=7]
    \arrow["\gamma", shorten <=2pt, shorten >=2pt, Rightarrow, nfold=3, from=9, to=11]
    \arrow["\delta", shorten <=2pt, shorten >=2pt, Rightarrow, nfold=3, from=11, to=10]
    \arrow["\gamma", shorten <=3pt, shorten >=3pt, Rightarrow, nfold=3, from=14, to=15]
    \arrow["\delta", shorten <=3pt, shorten >=3pt, Rightarrow, nfold=3, from=12, to=13]
    \arrow["\gamma", shorten <=2pt, shorten >=2pt, Rightarrow, nfold=3, from=16, to=17]
    \arrow["\delta", shorten <=2pt, shorten >=2pt, Rightarrow, nfold=3, from=18, to=19]
  \end{tikzcd}
\]

For every \(n\)-cell \(x\), there is an \((n+1)\)-cell \(\id(x) : x \to x\), called the \emph{identity morphism}.

As with 1-categories, \(\infty\)-categories need to satisfy certain equalities, which fall into 3 groups: associativity, unitality, and interchange. The associativity laws are the same as for 1-categories, only now a law is needed for each composition (every dimension and codimension).

Unitality is again similar to the case for 1-categories, except we again need unitality laws for each composition. We note that for lower codimension compositions, an iterated identity is needed. For example given a \(2\)-cell \(\alpha : f \to g\), the appropriate equation for left unitality of horizontal composition is:
\[ \id(\id(x)) \star_0 \alpha = \alpha \]
In general for a unit to be cancelled, it must be iterated a number of times equal to the difference between the dimension and codimension of the composition.

Interchange laws do not appear in 1-categories, and specify how compositions of different dimensions interact. The first interchange law states that for suitable \(2\)-cells \(\alpha\), \(\beta\), \(\gamma\), and \(\delta\), that:
\[ (\alpha \star_0 \gamma) \star_1 (\beta \star_0 \delta) = (\alpha \star_1 \beta) \star_0 ()\]
This can be diagrammatically depicted as:
\newsavebox{\innertop}
\savebox{\innertop}{
  \adjustbox{scale=0.8}{\begin{tikzcd}[ampersand replacement=\&,column sep=small]
    \bullet \& \bullet \& \bullet
    \arrow[""{name=0, anchor=center, inner sep=0}, curve={height=-12pt}, from=1-1, to=1-2]
    \arrow[""{name=1, anchor=center, inner sep=0}, curve={height=12pt}, from=1-1, to=1-2]
    \arrow[""{name=2, anchor=center, inner sep=0}, curve={height=-12pt}, from=1-2, to=1-3]
    \arrow[""{name=3, anchor=center, inner sep=0}, curve={height=12pt}, from=1-2, to=1-3]
    \arrow["\alpha", shorten <=3pt, shorten >=3pt, Rightarrow, nfold, from=1, to=0]
    \arrow["\gamma", shorten <=3pt, shorten >=3pt, Rightarrow, nfold, from=3, to=2]
  \end{tikzcd}}}
\newsavebox{\innerbot}
\savebox{\innerbot}{
  \adjustbox{scale=0.8}{\begin{tikzcd}[ampersand replacement=\&,column sep=small]
    \bullet \& \bullet \& \bullet
    \arrow[""{name=0, anchor=center, inner sep=0}, curve={height=-12pt}, from=1-1, to=1-2]
    \arrow[""{name=1, anchor=center, inner sep=0}, curve={height=12pt}, from=1-1, to=1-2]
    \arrow[""{name=2, anchor=center, inner sep=0}, curve={height=-12pt}, from=1-2, to=1-3]
    \arrow[""{name=3, anchor=center, inner sep=0}, curve={height=12pt}, from=1-2, to=1-3]
    \arrow["\beta", shorten <=3pt, shorten >=3pt, Rightarrow, nfold, from=1, to=0]
    \arrow["\delta", shorten <=3pt, shorten >=3pt, Rightarrow, nfold, from=3, to=2]
  \end{tikzcd}}}
\newsavebox{\innerleft}
\savebox{\innerleft}{
  \adjustbox{scale=1}{\begin{tikzcd}[ampersand replacement=\&,column sep=small,cramped]
    \bullet \& \bullet
    \arrow[""{name=0, anchor=center, inner sep=0}, controls=+(80:0.7) and +(100:0.7),, from=1-1, to=1-2]
    \arrow[""{name=1, anchor=center, inner sep=0}, curve={height=0}, from=1-1, to=1-2]
    \arrow[""{name=2, anchor=center, inner sep=0}, controls=+(100:-0.7) and +(80:-0.7),, from=1-1, to=1-2]
    \arrow["\alpha", shorten <=3pt, shorten >=3pt, Rightarrow, nfold, from=2, to=1]
    \arrow["\beta", shorten <=3pt, shorten >=3pt, Rightarrow, nfold, from=1, to=0]
  \end{tikzcd}}}
\newsavebox{\innerright}
\savebox{\innerright}{
  \adjustbox{scale=1}{\begin{tikzcd}[ampersand replacement=\&,column sep=small,cramped]
    \bullet \& \bullet
    \arrow[""{name=0, anchor=center, inner sep=0}, controls=+(80:0.7) and +(100:0.7), from=1-1, to=1-2]
    \arrow[""{name=1, anchor=center, inner sep=0}, curve={height=0}, from=1-1, to=1-2]
    \arrow[""{name=2, anchor=center, inner sep=0}, controls=+(100:-0.7) and +(80:-0.7),, from=1-1, to=1-2]
    \arrow["\gamma", shorten <=3pt, shorten >=3pt, Rightarrow, nfold, from=2, to=1]
    \arrow["\delta", shorten <=3pt, shorten >=3pt, Rightarrow, nfold, from=1, to=0]
  \end{tikzcd}}}
\[
  \begin{tikzcd}[column sep=small]
    \bullet &&&&& \bullet & {=} & \bullet &&& \bullet &&& \bullet
    \arrow[""{name=0, anchor=center, inner sep=0}, from=1-1, to=1-6]
    \arrow[""{name=1, anchor=center, inner sep=0}, draw=none, controls=+(90:2) and +(90:2), from=1-1, to=1-6]
    \arrow[""{name=2, anchor=center, inner sep=0}, draw=none, controls=+(90:-2) and +(90:-2), from=1-1, to=1-6]
    \arrow[""{name=4, anchor=center, inner sep=0}, draw=none, controls=+(80:1.5) and +(100:1.5), from=1-8, to=1-11]
    \arrow[""{name=5, anchor=center, inner sep=0}, draw=none, controls=+(100:-1.5) and +(80:-1.5), from=1-8, to=1-11]
    \arrow[""{name=6, anchor=center, inner sep=0}, draw=none, controls=+(80:1.5) and +(100:1.5), from=1-11, to=1-14]
    \arrow[""{name=8, anchor=center, inner sep=0}, draw=none, controls=+(100:-1.5) and +(80:-1.5), from=1-11, to=1-14]
    \arrow["\usebox{\innertop}"{description, inner sep = 0,xshift = -1.2pt}, shorten <=4pt, shorten >=4pt, Rightarrow, nfold, from=2, to=0]
    \arrow["\usebox{\innerbot}"{description, inner sep = 0,xshift = -1.2pt}, shorten <=4pt, shorten >=4pt, Rightarrow, nfold, from=0, to=1]
    \arrow[""{name=1, anchor=center, inner sep=0}, controls=+(90:2) and +(90:2), from=1-1, to=1-6]
    \arrow[""{name=2, anchor=center, inner sep=0}, controls=+(90:-2) and +(90:-2), from=1-1, to=1-6]
    \arrow["\usebox{\innerleft}"{description, inner sep = 0,xshift = -1.3pt}, shorten <=2pt, shorten >=2pt, Rightarrow, nfold, from=5, to=4]
    \arrow["\usebox{\innerright}"{description, inner sep = 0,xshift = -1.3pt}, shorten <=2pt, shorten >=2pt, Rightarrow, nfold, from=8, to=6]
    \arrow[controls=+(80:1.5) and +(100:1.5), from=1-8, to=1-11]
    \arrow[controls=+(100:-1.5) and +(80:-1.5), from=1-8, to=1-11]
    \arrow[controls=+(80:1.5) and +(100:1.5), from=1-11, to=1-14]
    \arrow[controls=+(100:-1.5) and +(80:-1.5), from=1-11, to=1-14]
  \end{tikzcd}
\]
% \[
%   \begin{tikzcd}[row sep = small]
%     \bullet & \bullet & \bullet \\
%     & {\star_1} \\
%     \bullet & \bullet & \bullet
%     \arrow[""{name=0, anchor=center, inner sep=0}, curve={height=-18pt}, from=1-1, to=1-2]
%     \arrow[""{name=1, anchor=center, inner sep=0}, curve={height=18pt}, from=1-1, to=1-2]
%     \arrow[""{name=2, anchor=center, inner sep=0}, curve={height=-18pt}, from=1-2, to=1-3]
%     \arrow[""{name=3, anchor=center, inner sep=0}, curve={height=18pt}, from=1-2, to=1-3]
%     \arrow[""{name=4, anchor=center, inner sep=0}, curve={height=-18pt}, from=3-1, to=3-2]
%     \arrow[""{name=5, anchor=center, inner sep=0}, curve={height=18pt}, from=3-1, to=3-2]
%     \arrow[""{name=6, anchor=center, inner sep=0}, curve={height=-18pt}, from=3-2, to=3-3]
%     \arrow[""{name=7, anchor=center, inner sep=0}, curve={height=18pt}, from=3-2, to=3-3]
%     \arrow["\alpha", shorten <=5pt, shorten >=5pt, Rightarrow, from=5, to=4]
%     \arrow["\gamma", shorten <=5pt, shorten >=5pt, Rightarrow, from=7, to=6]
%     \arrow["\beta", shorten <=5pt, shorten >=5pt, Rightarrow, from=1, to=0]
%     \arrow["\delta", shorten <=5pt, shorten >=5pt, Rightarrow, from=3, to=2]
%   \end{tikzcd}
%   \quad=\quad
%   \begin{tikzcd}
%     \bullet & \bullet
%     \arrow[""{name=0, anchor=center, inner sep=0}, curve={height=-30pt}, from=1-1, to=1-2]
%     \arrow[""{name=1, anchor=center, inner sep=0}, curve={height=30pt}, from=1-1, to=1-2]
%     \arrow[""{name=2, anchor=center, inner sep=0}, from=1-1, to=1-2]
%     \arrow["\alpha"', shorten <=4pt, shorten >=4pt, Rightarrow, from=1, to=2]
%     \arrow["\beta"', shorten <=4pt, shorten >=4pt, Rightarrow, from=2, to=0]
%   \end{tikzcd}
%   \star_0
%   \begin{tikzcd}
%     \bullet & \bullet
%     \arrow[""{name=0, anchor=center, inner sep=0}, curve={height=-30pt}, from=1-1, to=1-2]
%     \arrow[""{name=1, anchor=center, inner sep=0}, curve={height=30pt}, from=1-1, to=1-2]
%     \arrow[""{name=2, anchor=center, inner sep=0}, from=1-1, to=1-2]
%     \arrow["\gamma"', shorten <=4pt, shorten >=4pt, Rightarrow, from=1, to=2]
%     \arrow["\delta"', shorten <=4pt, shorten >=4pt, Rightarrow, from=2, to=0]
%   \end{tikzcd}
% \]
There are also interchange laws for the interaction of composition and identities; A composition of two identities is the same as an identity on the composition of the underlying cells.


The \(\infty\)-categories that we study in this thesis will be globular, meaning that their cells form a globular set. A globular set can be seen as natural extension of the data of a category, whose data can be arranged into the following diagram:
% https://q.uiver.app/#q=WzAsMixbMCwwLCJZIl0sWzEsMCwiWCJdLFswLDEsInMiLDAseyJvZmZzZXQiOi0xfV0sWzAsMSwidCIsMix7Im9mZnNldCI6MX1dXQ==
\[
  \begin{tikzcd}
    M & O
    \arrow["s", shift left, from=1-1, to=1-2]
    \arrow["t"', shift right, from=1-1, to=1-2]
  \end{tikzcd}
\]
where \(O\) is a set of objects, \(M\) is a set of all morphisms, and \(s\) and \(t\) are functions assigning each morphism to its source and target object respectively. \(2\)-cells can be added to this diagram in a natural way:

% https://q.uiver.app/#q=WzAsMyxbMSwwLCJDXzEiXSxbMiwwLCJDXzAiXSxbMCwwLCJDXzIiXSxbMCwxLCJzXzAiLDAseyJvZmZzZXQiOi0xfV0sWzAsMSwidF8wIiwyLHsib2Zmc2V0IjoxfV0sWzIsMCwic18xIiwwLHsib2Zmc2V0IjotMX1dLFsyLDAsInRfMSIsMix7Im9mZnNldCI6MX1dXQ==
\[
  \begin{tikzcd}
    {C_2} & {C_1} & {C_0}
    \arrow["{s_0}", shift left, from=1-2, to=1-3]
    \arrow["{t_0}"', shift right, from=1-2, to=1-3]
    \arrow["{s_1}", shift left, from=1-1, to=1-2]
    \arrow["{t_1}"', shift right, from=1-1, to=1-2]
  \end{tikzcd}
\]
In a globular set, the source and target of any cell must be parallel, meaning they share the same source and target. This condition is imposed by \emph{globularity conditions}. Adding these and iterating the process leads to the following definition.

\begin{definition}
  The category of globes \(\mathbf{G}\) has objects given by the natural numbers and morphisms generated from \(\mathbf{s}_n, \mathbf{t}_n : n \to n + 1\) quotiented by the \emph{globularity conditions}:
  \begin{align*}
    \mathbf{s}_{n+1} \circ \mathbf{s}_n &= \mathbf{t}_{n+1} \circ \mathbf{s}_n\\
    \mathbf{s}_{n+1} \circ \mathbf{t}_n &= \mathbf{t}_{n+1} \circ \mathbf{t}_n\\
  \end{align*}

   The category of globular sets \(\mathbf{Glob}\), is the presheaf category \([\mathbf{G}, \mathbf{Set}]\).
 \end{definition}

Unwrapping this definition, a globular set \(G\) consists of sets \(G(n)\) for each \(n \in \mathbb{N}\), with source and target maps \(s_n, t_n : G(n+1) \to G(n)\), forming the following diagram:
\[
  \begin{tikzcd}
    \cdots & {G(3)} & {G(2)} & {G(1)} & {G(0)}
    \arrow["{s_0}", shift left, from=1-4, to=1-5]
    \arrow["{t_0}"', shift right, from=1-4, to=1-5]
    \arrow["{s_1}", shift left, from=1-3, to=1-4]
    \arrow["{t_1}"', shift right, from=1-3, to=1-4]
    \arrow["{t_2}"', shift right, from=1-2, to=1-3]
    \arrow["{s_2}", shift left, from=1-2, to=1-3]
    \arrow[shift right, from=1-1, to=1-2]
    \arrow[shift left, from=1-1, to=1-2]
  \end{tikzcd}
\]
and satisfying the globularity conditions. A morphism of globular sets \(F : G \to H\) is a collection of functions \(G(n) \to H(n)\) which commute with source and target maps.

Given a globular set \(G\), we will call the elements of \(G(n)\) the \(n\)-cells and write \(f : x \to y\) for an \((n+1)\)-cell \(f\) where \(s_n(f) = x\) and \(t_n(f) = y\). Further we will drop the subscripts on \(s_n\) and \(t_n\) when they are clear and write \(s^n\) and \(t^n\) to denote \(s\) or \(t\) applied \(n\) times.\todo{Add coinductive definition?}

\begin{example}
  The \(n\)-disc \(D^n\) is a finite globular set given by \(Y(n)\), where \(Y\) is the Yoneda functor \(\mathbf{G} \to \mathbf{Glob}\). \(D^n\), has no \(k\)-cells for \(k > n\), a single \(n\)-cell \(d_n\), and two \(m\)-cells \(d_m^-\) and \(d_m^+\) for \(m < n\). Every \((m+1)\)-cell of \(D^n\) has source \(d_m^-\) and target \(d_m^+\). The first few discs are depicted in \cref{fig:discs}. The Yoneda lemma tells us that a map of globular sets \(D^n \to G\) is the same as an \(n\)-cell of \(G\) (in fact the \(n\)-cell which \(d_n\) is sent to).
\end{example}

\begin{figure}[h]
  \centering
  \begin{tabular}{P{3cm} P{3cm} P{3cm} P{3cm}}
    \(D^0\)&\(D^1\)&\(D^2\)&\(D^3\)\\
    {\begin{tikzcd}
        d_0
      \end{tikzcd}
    }&{\begin{tikzcd}[ampersand replacement=\&]
        d_0^- \& d_0^+
        \arrow[from=1-1, to=1-2, "d_1"]
      \end{tikzcd}
       }&{\begin{tikzcd}[ampersand replacement=\&]
           d_0^- \& d_0^+
           \arrow[""{name=0, anchor=center, inner sep=0}, "d_1^+", curve={height=-18pt}, from=1-1, to=1-2]
           \arrow[""{name=1, anchor=center, inner sep=0}, "d_1^-"', curve={height=18pt}, from=1-1, to=1-2]
           \arrow["d_2", shorten <=3pt, shorten >=3pt, Rightarrow, from=1, to=0]
         \end{tikzcd}
          }&{\begin{tikzcd}[ampersand replacement=\&]
              d_0^- \&\& d_0^+
              \arrow[""{name=0, anchor=center, inner sep=0}, "d_1^+", curve={height=-25pt}, from=1-1, to=1-3]
              \arrow[""{name=1, anchor=center, inner sep=0}, "d_1^-"', curve={height=25pt}, from=1-1, to=1-3]
              \arrow[""{name=2, anchor=center, inner sep=0}, "d_2^-", shift left=12pt,Rightarrow, nfold, shorten <=5pt, shorten >=5pt, from=1,to=0]
              \arrow[""{name=3, anchor=center, inner sep=0}, "d_2^+"', shift right=12pt,Rightarrow, nfold, shorten <=5pt, shorten >=5pt, from=1,to=0]
              \arrow["d_3", Rightarrow, nfold = 3, shorten <=3pt, shorten >=3pt,from=2,to=3]
            \end{tikzcd}}
  \end{tabular}
  \caption{The first disc globular sets}
  \label{fig:discs}
\end{figure}

We can now give the definition of a (strict) \(\infty\)-category.

\begin{definition}
  A (strict) \(\infty\)-category is a globular set \(G\) with operations:
  \begin{itemize}
  \item For \(m < n\), a composition \(\star_m\) taking \(n\)-cells \(f\) and \(g\) with \(t^{n-m}(f) = s^{n-m}(g)\) and giving an \(n\)-cell \(f \star_m g\) with:
    \begin{align*}
      s(f \star_m g) &= \begin{cases*}
        s(f)&\text{if \(m = n - 1\)}\\
        s(f) \star_m s(g)&\text{otherwise}
      \end{cases*}\\
      t(f \star_m g) &= \begin{cases*}
        t(g)&\text{if \(m = n - 1\)}\\
        t(f) \star_m t(g)&\text{otherwise}
      \end{cases*}
    \end{align*}
  \item For \(n\)-cell \(x\), an identity \((n+1)\)-cell \(\id(x) : x \to x\).
  \end{itemize}
  and satisfying equalities:
  \begin{itemize}
  \item Associativity: Given \(m < n\) and \(n\)-cells \(f\), \(g\), and \(h\) with \(t^{n-m}(f) = s^{n-m}(g)\) and \(t^{n-m}(g) = s^{n-m}(h)\):
    \[ (f \star_m g) \star_m h = f \star_m (g \star_m h) \]
  \item Unitality: Given \(m < n\) and \(n\)-cell \(f\):
    \begin{align*}
      \id(s^{n-m}(f)) \star_m f &= f\\
      f \star_m \id(t^{n-m}(f)) &= f
    \end{align*}
  \item Composition interchange: If \(o < m < n\) and \(\alpha\), \(\beta\), \(\gamma\), and \(\delta\) be \(n\)-cells with
    \[t^{n-m}(\alpha) = s^{n-m}(\beta)\qquad
      t^{n-m}(\gamma) = s^{n-m}(\delta)\qquad
      t^{n-o}(\alpha) = s^{n-o}(\gamma)\]
    then:
    \[(\alpha \star_o \gamma) \star_m (\beta \star_o \delta) = (\alpha \star_m \beta) \star_o (\gamma \star_m \delta)\]
  \item Identity interchange: Let \(m < n\) and \(f\) and \(g\) be \(n\)-cells with \(t^{n-m}(f) = s^{n-m}(g)\). Then:
    \[\id(f) \star_m \id(g) = \id(f \star_m g)\]
  \end{itemize}
  A morphism of \(\infty\) categories is a morphism of the underlying globular sets which preserves composition and identities.
\end{definition}

There is a clear forgetful functor from the category of strict infinity categories to the category of globular sets, which has a left adjoint given by taking the free strict infinity category over a globular set. \todo{Do i need monad?}

We end this section with an example of a non-trivial application of the axioms of an infinity category, known as the Eckmann-Hilton argument. The argument show's that any two scalars (morphisms from the identity to the identity) commute.

\begin{proposition}[Eckmann-Hilton]
  Let \(x\) be an \(n\)-cell in an \(\infty\)-category and let \(\alpha\) and \(\beta\) be \((n+2)\)-cells with source and target \(\id(x)\). Then \(\alpha \star_{n+1} \beta = \beta \star_{n+1} \alpha\).
\end{proposition}
\begin{proof}
  The cells \(\alpha\) and \(\beta\) can be manoeuvred around each other as follows:
  \begin{align*}
    &\phantom{{}={}} \alpha \star_{n+1} \beta \\
    &= (\alpha \star_n i) \star_{n+1} (i \star_n \beta)&\text{Unitality}\\
    &= (\alpha \star_{n+1} i) \star_n (i \star_{n+1} \beta)&\text{Interchange}\\
    &= \alpha \star_n \beta &\text{Unitality}\\
    &= (i \star_{n+1} \alpha) \star_n (\beta \star_{n+1} i)&\text{Unitality}\\
    &= (i \star_n \beta) \star_{n+1} (\alpha \star_n i)&\text{Interchange}\\
    &= \beta \star_{n+1} \alpha&\text{Unitality}
  \end{align*}
  Where \(i = \id(\id(x))\).
\end{proof}

We give a more graphical representation of the proof in \cref{fig:eh}. In this proof the \(\alpha\) is moved to the left of \(\beta\), though we equally could have moved it round the right, and the choice made was arbitrary.

\newsavebox{\ehalpha}
\savebox{\ehalpha}{\adjustbox{scale=0.8}{
  \begin{tikzcd}[ampersand replacement=\&,column sep=small,cramped]
    \bullet \& \bullet \& \bullet
    \arrow[""{name=0, anchor=center, inner sep=0}, curve={height=-10pt}, from=1-1, to=1-2]
    \arrow[""{name=1, anchor=center, inner sep=0}, curve={height=10pt}, from=1-1, to=1-2]
    \arrow[""{name=2, anchor=center, inner sep=0}, curve={height=-10pt}, from=1-2, to=1-3]
    \arrow[""{name=3, anchor=center, inner sep=0}, curve={height=10pt}, from=1-2, to=1-3]
    \arrow["\alpha"', color={rgb,255:red,0;green,24;blue,204}, shorten <=3pt, shorten >=3pt, Rightarrow, nfold, from=1, to=0]
    \arrow["\id"', shorten <=3pt, shorten >=3pt, Rightarrow, nfold, from=3, to=2]
  \end{tikzcd}}}
\newsavebox{\ehbeta}
\savebox{\ehbeta}{\adjustbox{scale=0.8}{
  \begin{tikzcd}[ampersand replacement=\&,column sep=small,cramped]
    \bullet \& \bullet \& \bullet
    \arrow[""{name=0, anchor=center, inner sep=0}, curve={height=-10pt}, from=1-1, to=1-2]
    \arrow[""{name=1, anchor=center, inner sep=0}, curve={height=10pt}, from=1-1, to=1-2]
    \arrow[""{name=2, anchor=center, inner sep=0}, curve={height=-10pt}, from=1-2, to=1-3]
    \arrow[""{name=3, anchor=center, inner sep=0}, curve={height=10pt}, from=1-2, to=1-3]
    \arrow["\id"',  shorten <=3pt, shorten >=3pt, Rightarrow, nfold, from=1, to=0]
    \arrow["\beta"', color={rgb,255:red,204;green,0;blue,14}, shorten <=3pt, shorten >=3pt, Rightarrow, nfold, from=3, to=2]
  \end{tikzcd}}}
\newsavebox{\ehlefttop}
\savebox{\ehlefttop}{
  \adjustbox{scale=1}{\begin{tikzcd}[ampersand replacement=\&,column sep=small,cramped]
     \bullet \& \bullet
     \arrow[""{name=0, anchor=center, inner sep=0}, controls=+(80:0.7) and +(100:0.7),, from=1-1, to=1-2]
     \arrow[""{name=1, anchor=center, inner sep=0}, curve={height=0}, from=1-1, to=1-2]
     \arrow[""{name=2, anchor=center, inner sep=0}, controls=+(100:-0.7) and +(80:-0.7),, from=1-1, to=1-2]
     \arrow["\alpha", color= blue, shorten <=3pt, shorten >=3pt, Rightarrow, nfold, from=2, to=1]
     \arrow["\id", shorten <=3pt, shorten >=3pt, Rightarrow, nfold, from=1, to=0]
   \end{tikzcd}}}
\newsavebox{\ehrighttop}
\savebox{\ehrighttop}{
  \adjustbox{scale=1}{\begin{tikzcd}[ampersand replacement=\&,column sep=small,cramped]
    \bullet \& \bullet
    \arrow[""{name=0, anchor=center, inner sep=0}, controls=+(80:0.7) and +(100:0.7), from=1-1, to=1-2]
    \arrow[""{name=1, anchor=center, inner sep=0}, curve={height=0}, from=1-1, to=1-2]
    \arrow[""{name=2, anchor=center, inner sep=0}, controls=+(100:-0.7) and +(80:-0.7),, from=1-1, to=1-2]
    \arrow["\id", shorten <=3pt, shorten >=3pt, Rightarrow, nfold, from=2, to=1]
    \arrow["\beta", color=red, shorten <=3pt, shorten >=3pt, Rightarrow, nfold, from=1, to=0]
  \end{tikzcd}}}
\newsavebox{\ehleftbot}
\savebox{\ehleftbot}{
  \adjustbox{scale=1}{\begin{tikzcd}[ampersand replacement=\&,column sep=small,cramped]
    \bullet \& \bullet
    \arrow[""{name=0, anchor=center, inner sep=0}, controls=+(80:0.7) and +(100:0.7),, from=1-1, to=1-2]
    \arrow[""{name=1, anchor=center, inner sep=0}, curve={height=0}, from=1-1, to=1-2]
    \arrow[""{name=2, anchor=center, inner sep=0}, controls=+(100:-0.7) and +(80:-0.7),, from=1-1, to=1-2]
    \arrow["\id", shorten <=3pt, shorten >=3pt, Rightarrow, nfold, from=2, to=1]
    \arrow["\alpha", color=blue, shorten <=3pt, shorten >=3pt, Rightarrow, nfold, from=1, to=0]
  \end{tikzcd}}}
\newsavebox{\ehrightbot}
\savebox{\ehrightbot}{
  \adjustbox{scale=1}{\begin{tikzcd}[ampersand replacement=\&,column sep=small,cramped]
    \bullet \& \bullet
    \arrow[""{name=0, anchor=center, inner sep=0}, controls=+(80:0.7) and +(100:0.7), from=1-1, to=1-2]
    \arrow[""{name=1, anchor=center, inner sep=0}, curve={height=0}, from=1-1, to=1-2]
    \arrow[""{name=2, anchor=center, inner sep=0}, controls=+(100:-0.7) and +(80:-0.7),, from=1-1, to=1-2]
    \arrow["\beta", color=red, shorten <=3pt, shorten >=3pt, Rightarrow, nfold, from=2, to=1]
    \arrow["\id", shorten <=3pt, shorten >=3pt, Rightarrow, nfold, from=1, to=0]
  \end{tikzcd}}}

\begin{figure}[h]
  \centering

  \[
    \begin{tikzcd}[ampersand replacement=\&,column sep=small]
      \bullet \&\& \bullet \& = \& \bullet \&\&\&\&\& \bullet \& = \& \bullet \&\&\& \bullet \&\&\& \bullet \\
      \\
      \&\&\&\&\&\&\&\&\&\&\&\&\&\& = \\
      \\
      \bullet \&\& \bullet \& = \& \bullet \&\&\&\&\& \bullet \& = \& \bullet \&\&\& \bullet \&\&\& \bullet
      \arrow[""{name=0, anchor=center, inner sep=0}, "\id", curve={height=-24pt}, from=1-1, to=1-3]
      \arrow[""{name=1, anchor=center, inner sep=0}, "\id"', curve={height=24pt}, from=1-1, to=1-3]
      \arrow[""{name=2, anchor=center, inner sep=0}, "\id"{description}, from=1-1, to=1-3]
      \arrow[""{name=3, anchor=center, inner sep=0}, draw=none, controls=+(90:1.8) and +(90:1.8), from=1-5, to=1-10]
      \arrow[""{name=4, anchor=center, inner sep=0}, draw=none, controls=+(90:-1.8) and +(90:-1.8), from=1-5, to=1-10]
      \arrow[""{name=5, anchor=center, inner sep=0}, from=1-5, to=1-10]
      \arrow[""{name=6, anchor=center, inner sep=0}, draw=none, controls=+(90:1.8) and +(90:1.8), from=5-5, to=5-10]
      \arrow[""{name=7, anchor=center, inner sep=0}, draw=none, controls=+(90:-1.8) and +(90:-1.8), from=5-5, to=5-10]
      \arrow[""{name=8, anchor=center, inner sep=0}, from=5-5, to=5-10]
      \arrow[""{name=9, anchor=center, inner sep=0}, "\id", curve={height=-24pt}, from=5-1, to=5-3]
      \arrow[""{name=10, anchor=center, inner sep=0}, "\id"', curve={height=24pt}, from=5-1, to=5-3]
      \arrow[""{name=11, anchor=center, inner sep=0}, "\id"{description}, from=5-1, to=5-3]
      \arrow[""{name=12, anchor=center, inner sep=0}, draw=none, controls=+(80:1.5) and +(100:1.5), from=1-12, to=1-15]
      \arrow[""{name=13, anchor=center, inner sep=0}, draw=none, controls=+(100:-1.5) and +(80:-1.5), from=1-12, to=1-15]
      \arrow[""{name=14, anchor=center, inner sep=0}, draw=none, controls=+(80:1.5) and +(100:1.5), from=1-15, to=1-18]
      \arrow[""{name=15, anchor=center, inner sep=0}, draw=none, controls=+(100:-1.5) and +(80:-1.5), from=1-15, to=1-18]
      \arrow[""{name=16, anchor=center, inner sep=0}, draw=none, controls=+(80:1.5) and +(100:1.5), from=5-12, to=5-15]
      \arrow[""{name=17, anchor=center, inner sep=0}, draw=none, controls=+(100:-1.5) and +(80:-1.5), from=5-12, to=5-15]
      \arrow[""{name=18, anchor=center, inner sep=0}, draw=none, controls=+(80:1.5) and +(100:1.5), from=5-15, to=5-18]
      \arrow[""{name=19, anchor=center, inner sep=0}, draw=none, controls=+(100:-1.5) and +(80:-1.5), from=5-15, to=5-18]
      \arrow["\alpha"', color={rgb,255:red,0;green,24;blue,204}, shorten <=3pt, shorten >=5pt, Rightarrow, nfold, from=1, to=2]
      \arrow["\beta"', color={rgb,255:red,204;green,0;blue,14}, shorten <=5pt, shorten >=3pt, Rightarrow, nfold, from=2, to=0]
      \arrow["\beta"', color={rgb,255:red,204;green,0;blue,14}, shorten <=3pt, shorten >=5pt, Rightarrow, nfold, from=10, to=11]
      \arrow["\alpha"', color={rgb,255:red,0;green,24;blue,204}, shorten <=5pt, shorten >=3pt, Rightarrow, nfold, from=11, to=9]
      \arrow["\usebox{\ehalpha}"{description,inner sep = 0,xshift = -1.2pt}, shorten <=3pt, shorten >=3pt, Rightarrow, nfold, from=4, to=5]
      \arrow["\usebox{\ehbeta}"{description,inner sep = 0,xshift = -1.2pt}, shorten <=3pt, shorten >=3pt, Rightarrow, nfold, from=5, to=3]
      \arrow["\usebox{\ehbeta}"{description,inner sep = 0,xshift = -1.2pt}, shorten <=3pt, shorten >=3pt, Rightarrow, nfold, from=7, to=8]
      \arrow["\usebox{\ehalpha}"{description,inner sep = 0,xshift = -1.2pt}, shorten <=3pt, shorten >=3pt, Rightarrow, nfold, from=8, to=6]
      \arrow["\usebox{\ehlefttop}"{description,inner sep = 0,xshift = -1.3pt, yshift = 0.2pt}, shorten <=3pt, shorten >=3pt, Rightarrow, nfold, from=13, to=12]
      \arrow["\usebox{\ehrighttop}"{description,inner sep = 0,xshift = -1.3pt,yshift = 0.2pt}, shorten <=3pt, shorten >=3pt, Rightarrow, nfold, from=15, to=14]
      \arrow["\usebox{\ehleftbot}"{description,inner sep = 0,xshift = -1.3pt}, shorten <=3pt, shorten >=3pt, Rightarrow, nfold, from=17, to=16]
      \arrow["\usebox{\ehrightbot}"{description,inner sep = 0,xshift = -1.3pt}, shorten <=3pt, shorten >=3pt, Rightarrow, nfold, from=19, to=18]
      \arrow[controls=+(90:1.8) and +(90:1.8), from=1-5, to=1-10]
      \arrow[controls=+(90:-1.8) and +(90:-1.8), from=1-5, to=1-10]
      \arrow[controls=+(90:1.8) and +(90:1.8), from=5-5, to=5-10]
      \arrow[controls=+(90:-1.8) and +(90:-1.8), from=5-5, to=5-10]
      \arrow[controls=+(80:1.5) and +(100:1.5), from=1-12, to=1-15]
      \arrow[controls=+(100:-1.5) and +(80:-1.5), from=1-12, to=1-15]
      \arrow[controls=+(80:1.5) and +(100:1.5), from=1-15, to=1-18]
      \arrow[controls=+(100:-1.5) and +(80:-1.5), from=1-15, to=1-18]
      \arrow[controls=+(80:1.5) and +(100:1.5), from=5-12, to=5-15]
      \arrow[controls=+(100:-1.5) and +(80:-1.5), from=5-12, to=5-15]
      \arrow[controls=+(80:1.5) and +(100:1.5), from=5-15, to=5-18]
      \arrow[controls=+(100:-1.5) and +(80:-1.5), from=5-15, to=5-18]
    \end{tikzcd}
  \]
  \caption{The Eckmann-Hilton argument}
  \label{fig:eh}
\end{figure}

\subsection{Pasting schemes}
\label{sec:pasting-schemes}



\subsection{Strictness}
\label{sec:strictness}

The \(\infty\)-categories we have defined so far are known as \emph{strict} infinity categories.

\todo[inline]{Isomorphism vs equality}

\todo[inline]{Topological example}

\todo[inline]{Smaller monoidal category example}

\todo[inline]{Definition of monoidal category}

\todo[inline]{Strictification for monoidal categories}

\todo[inline]{All works for 2-categories}

\todo[inline]{Eckmann-Hilton in weak setting}

\todo[inline]{Braiding and counter example for 3-dimension semistrictification.}

\todo[inline]{Exponential blowup in number of coherence moves needed}

\todo[inline]{Informal description of weak infinity categories}

\todo[inline]{Conjectures}

\section{Type theory preliminaries}
\label{sec:type-theory}

\todo[inline]{Background on type theory}

\todo[inline]{Dependent type theory}

\todo[inline]{MLTT/HoTT}

\todo[inline]{Properties of type theories}

\section{The type theory \Catt}
\label{sec:type-theory-catt}

\todo[inline]{High level presentation of Catt}

\todo[inline]{We give a more formal definition in the next section}

\todo[inline]{Talk about models of Catt here}

\chapter{A general presentation of \Catt}
\label{cha:gener-pres-catt}

\todo[inline]{Should be able take this mainly from cattsua paper}

\section{\Catt with equality}
\label{sec:catt-with-equality}

\section{Properties of \Catt with equality}
\label{sec:properties-catt-with}

\section{Support conditions}
\label{sec:support-conditions}

\todo[inline]{Ideas about invertibility here?}

\chapter{Constructions in \Catt}
\label{sec:operations-catt}

\section{Basic constructions}
\label{sec:basic-constructions}

\section{Structured terms}
\label{sec:structured-terms}

\section{Pruning}
\label{sec:pruning}

\section{Insertion}
\label{sec:insertion}

\chapter{Semistrict versions of \Catt}
\label{cha:cattsu}

\section{\Cattsu}
\label{sec:cattsu}

\section{\Cattsua}
\label{sec:cattsua}

\section{Rehydration}
\label{sec:rehydration}

\todo[inline]{Explain original motivation}

\todo[inline]{Explain connection to equivalence}

\todo[inline]{core problem with rehydration}

\subsection{Rehydration by dimension}
\label{sec:rehydr-dimens}

\subsection{Rehydration by cylinders}
\label{sec:rehydr-cylind}


















%%%%%%%%%%%%%%%%%%%%%%%%%%%%%%%%%%%%%%%%%%%%%%%%%%%%%%%%%%%%%%%%%%%%%%%%%%%%%%%%
%% References:
%%

\printbibliography

\end{document}
