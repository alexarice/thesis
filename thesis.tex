\documentclass{cam-thesis}

\usepackage{packages}
\usepackage{macros}


\title{A type-theoretic approach to semistrict higher categories}

%% The full name of the author (e.g.: James Smith):
\author{Alex Rice}

%% College affiliation:
\college{Darwin College}

%% College shield:
\collegeshield{CollegeShields/Darwin}

%% Submission date [optional]:
\submissiondate{TODO}

%% Declaration date:
\date{TODO}

%% PDF meta-info:
\subjectline{Computer Science}
\keywords{category theory, higher category theory, type theory}



%%%%%%%%%%%%%%%%%%%%%%%%%%%%%%%%%%%%%%%%%%%%%%%%%%%%%%%%%%%%%%%%%%%%%%%%%%%%%%%%
%% Abstract:
%%
\abstract{%
  Abstract to go here...
}


%%%%%%%%%%%%%%%%%%%%%%%%%%%%%%%%%%%%%%%%%%%%%%%%%%%%%%%%%%%%%%%%%%%%%%%%%%%%%%%%
%% Acknowledgements:
%%
\acknowledgements{%
  My acknowledgements ...
}


%%%%%%%%%%%%%%%%%%%%%%%%%%%%%%%%%%%%%%%%%%%%%%%%%%%%%%%%%%%%%%%%%%%%%%%%%%%%%%%%
%% Contents:
%%
\begin{document}

%%%%%%%%%%%%%%%%%%%%%%%%%%%%%%%%%%%%%%%%%%%%%%%%%%%%%%%%%%%%%%%%%%%%%%%%%%%%%%%%
%% Title page, abstract, declaration etc.:
%% -    the title page (is automatically omitted in the technical report mode).
\frontmatter{}



%%%%%%%%%%%%%%%%%%%%%%%%%%%%%%%%%%%%%%%%%%%%%%%%%%%%%%%%%%%%%%%%%%%%%%%%%%%%%%%%
%% Thesis body:
%%
\chapter{Introduction}

\section{Higher categories}
\label{sec:higher-categories}

A higher category is a generalisation of the ordinary notion of a category to allow higher dimensional structure. This manifests in the form of allowing arrows or morphisms to have their source or target be another morphism instead of an object. More precisely, higher categories are equipped with the notion of an \(n\)-cell, where an \((n+1)\)-cell has source and target \(n\)-cells, and \(0\)-cells play the role of objects in an ordinary category.
\todo[inline]{Higher categories through pictures}

\todo[inline]{Composition}

\todo[inline]{Example of 2-categories with interchange}

One of the simplest examples of a higher category is a 2-category. These have just \(0\), \(1\), and \(2\)-cells, and no structure higher than this.\todo{remove this}

\begin{definition}
  A (strict) \emph{2-category} consists of:
  \begin{itemize}
  \item A set of objects (or \(0\)-cells).
  \item For every pair of \(0\) cells \(x, y \in X\), a set of \(1\)-cells. We write \(f \colon x \to y\) to say that \(f\) is in this set and say that \(f\) has source \(x\) and target \(y\). We may draw this as:
    \[
      \begin{tikzcd}
	x & y
	\arrow["f", from=1-1, to=1-2]
      \end{tikzcd}
    \]
    Two \(1\)-cells are \emph{parallel} if they have the same source and target.
  \item For every parallel pair of \(1\)-cells \(f,g \colon x \to y\), a set of \(2\)-cells. Elements of this set are written \(\alpha \colon f \to g\), similar to \(1\)-cells, and may be drawn:

    \[\begin{tikzcd}
	x & y
	\arrow[""{name=0, anchor=center, inner sep=0}, "f", curve={height=-12pt}, from=1-1, to=1-2]
	\arrow[""{name=1, anchor=center, inner sep=0}, "g"', curve={height=12pt}, from=1-1, to=1-2]
	\arrow["\alpha", shorten <=3pt, shorten >=3pt, Rightarrow, from=0, to=1]
      \end{tikzcd}
    \]
  \item For each \(0\)-cell \(x\), an identity \(1\)-cell \(\iden{x} \colon x \to x\).
  \item For each \(1\)-cell \(f\), an identity \(0\)-cell \(\iden{f} \colon f \to f\).
  \item For \(1\)-cells \(f \colon x \to y\) and \(g \colon y \to z\), a composition \(g \circ f \colon x \to z\).
  \item For \(2\)-cells \(\alpha \colon f \to g\) and \(\beta \colon g \to h\), a vertical composition \(\beta \circ_v \alpha \colon f \to h\).
  \item
  \end{itemize}
\end{definition}\todo{remove this}

\todo[inline]{Example: Category of categories}

\todo[inline]{Formal definition of globular set}

\todo[inline]{Note on coinductive definitions}

\todo[inline]{Definition of strict \(\infty\)-category}

\subsection{Strictness}
\label{sec:strictness}

The \(\infty\)-categories we have defined so far are known as \emph{strict} infinity categories.

\section{Type theory preliminaries}
\label{sec:type-theory}

\subsection{Models of type theory}
\label{sec:models-type-theory}

\subsection{Formalisation}
\label{sec:formalisation}

\section{The type theory \Catt}
\label{sec:type-theory-catt}

\chapter{A general presentation of \Catt}
\label{cha:gener-pres-catt}

\section{\Catt with equality}
\label{sec:catt-with-equality}

\section{Support conditions}
\label{sec:support-conditions}

\section{Properties of \Catt with equality}
\label{sec:properties-catt-with}

\chapter{Constructions in \Catt}
\label{sec:operations-catt}

\section{Basic constructions}
\label{sec:basic-constructions}

\section{Structured terms}
\label{sec:structured-terms}

\section{Pruning}
\label{sec:pruning}

\section{Insertion}
\label{sec:insertion}

\chapter{\Cattsu : \Catt with strict units}
\label{cha:cattsu}

\chapter{\Cattsa : \Catt with strict associativity}
\label{cha:cattsa}

\chapter{\Cattsua : \Catt with strict units and associativity}
\label{cha:cattsua}

















%%%%%%%%%%%%%%%%%%%%%%%%%%%%%%%%%%%%%%%%%%%%%%%%%%%%%%%%%%%%%%%%%%%%%%%%%%%%%%%%
%% References:
%%

\printbibliography

\end{document}
