\documentclass{cam-thesis}

\usepackage{packages}
\usepackage{macros}


\title{A type-theoretic approach to semistrict higher categories}

%% The full name of the author (e.g.: James Smith):
\author{Alex Rice}

%% College affiliation:
\college{Darwin College}

%% College shield:
\collegeshield{CollegeShields/Darwin}

%% Submission date [optional]:
\submissiondate{TODO}

%% Declaration date:
\date{TODO}

%% PDF meta-info:
\subjectline{Computer Science}
\keywords{category theory, higher category theory, type theory}



%%%%%%%%%%%%%%%%%%%%%%%%%%%%%%%%%%%%%%%%%%%%%%%%%%%%%%%%%%%%%%%%%%%%%%%%%%%%%%%%
%% Abstract:
%%
\abstract{%
  Abstract to go here...
}


%%%%%%%%%%%%%%%%%%%%%%%%%%%%%%%%%%%%%%%%%%%%%%%%%%%%%%%%%%%%%%%%%%%%%%%%%%%%%%%%
%% Acknowledgements:
%%
\acknowledgements{%
  My acknowledgements ...
}


%%%%%%%%%%%%%%%%%%%%%%%%%%%%%%%%%%%%%%%%%%%%%%%%%%%%%%%%%%%%%%%%%%%%%%%%%%%%%%%%
%% Contents:
%%
\begin{document}

%%%%%%%%%%%%%%%%%%%%%%%%%%%%%%%%%%%%%%%%%%%%%%%%%%%%%%%%%%%%%%%%%%%%%%%%%%%%%%%%
%% Title page, abstract, declaration etc.:
%% -    the title page (is automatically omitted in the technical report mode).
\frontmatter{}



%%%%%%%%%%%%%%%%%%%%%%%%%%%%%%%%%%%%%%%%%%%%%%%%%%%%%%%%%%%%%%%%%%%%%%%%%%%%%%%%
%% Thesis body:
%%
\chapter{Introduction}

\todo[inline] {Write intro here}

\section{Higher categories}
\label{sec:higher-categories}

A higher category is a generalisation of the ordinary notion of a category to allow higher dimensional structure. This manifests in the form of allowing arrows or morphisms to have their source or target be another morphism instead of an object. More precisely, higher categories are equipped with the notion of an \(n\)-cell, where an \((n+1)\)-cell has source and target \(n\)-cells, and \(0\)-cells play the role of objects in an ordinary category.

\todo[inline]{Introduce Globular sets, use pictures of arrows}

\todo[inline]{Composition, with pictures}

\todo[inline]{Definition of strict \(\infty\)-category}

\todo[inline]{Note on coinductive definitions}

\subsection{Strictness}
\label{sec:strictness}

The \(\infty\)-categories we have defined so far are known as \emph{strict} infinity categories.

\todo[inline]{Talk about isomorphism vs equality}

\todo[inline]{Define bicategory?}

\todo[inline]{Exponential blowup in number of coherence moves needed}

\section{Type theory preliminaries}
\label{sec:type-theory}

\todo[inline]{Background on type theory}

\todo[inline]{Dependent type theory}

\todo[inline]{MLTT/HoTT}

\todo[inline]{Properties of type theories}

\section{The type theory \Catt}
\label{sec:type-theory-catt}

\todo[inline]{High level presentation of Catt}

\todo[inline]{We give a more formal definition in the next section}

\todo[inline]{Talk about models of Catt here}

\subsection{Formalisation}
\label{sec:formalisation}

\chapter{A general presentation of \Catt}
\label{cha:gener-pres-catt}

\todo[inline]{Should be able take this mainly from cattsua paper}

\section{\Catt with equality}
\label{sec:catt-with-equality}

\section{Properties of \Catt with equality}
\label{sec:properties-catt-with}

\section{Support conditions}
\label{sec:support-conditions}

\todo[inline]{Ideas about invertibility here?}

\chapter{Constructions in \Catt}
\label{sec:operations-catt}

\section{Basic constructions}
\label{sec:basic-constructions}

\section{Structured terms}
\label{sec:structured-terms}

\section{Pruning}
\label{sec:pruning}

\section{Insertion}
\label{sec:insertion}

\chapter{Semistrict versions of \Catt}
\label{cha:cattsu}

\subsection{\Cattsu}
\label{sec:cattsu}

\subsection{\Cattsua}
\label{sec:cattsua}

\subsection{Rehydration}
\label{sec:rehydration}

\todo[inline]{Explain original motivation}

\todo[inline]{Explain connection to equivalence}

\todo[inline]{core problem with rehydration}

\subsubsection{Rehydration by dimension}
\label{sec:rehydr-dimens}

\subsubsection{Rehydration by cylinders}
\label{sec:rehydr-cylind}


















%%%%%%%%%%%%%%%%%%%%%%%%%%%%%%%%%%%%%%%%%%%%%%%%%%%%%%%%%%%%%%%%%%%%%%%%%%%%%%%%
%% References:
%%

\printbibliography

\end{document}
